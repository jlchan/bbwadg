\pdfoutput=1

\documentclass{article}
\usepackage{fullpage}
\usepackage{amsmath,amssymb,amsfonts}
\usepackage{bm}
%\usepackage{bbm}
%\usepackage{tikz}
%\usepackage[normalem]{ulem}
%\usepackage{hhline}
%\usepackage{graphicx}
%\usepackage{subfig}
\usepackage{color}
%\usepackage{doi}

%\renewcommand{\topfraction}{0.85}
%\renewcommand{\textfraction}{0.1}
%\renewcommand{\floatpagefraction}{0.75}
%
%
%\newcommand{\bbm}[1]{\mathbbm{#1}}
%\newcommand{\bs}[1]{\boldsymbol{#1}}
%\newcommand{\equaldef}{\stackrel{\mathrm{def}}{=}}
%
%
%\newcommand{\mb}[1]{\mathbf{#1}}
%\newcommand{\mbb}[1]{\mathbb{#1}}
%\newcommand{\mc}[1]{\mathcal{#1}}
%
%\renewcommand{\hat}{\widehat}
%\renewcommand{\tilde}{\widetilde}
\newcommand{\td}[2]{\frac{{\rm d}#1}{{\rm d}{\rm #2}}}
\newcommand{\pd}[2]{\frac{\partial#1}{\partial#2}}
%\newcommand{\pdn}[3]{\frac{\partial^{#3}#1}{\partial#2^{#3}}}
%\newcommand{\snor}[1]{\left| #1 \right|}
%\newcommand{\nor}[1]{\left\| #1 \right\|}
\newcommand{\LRp}[1]{\left( #1 \right)}
\newcommand{\LRs}[1]{\left[ #1 \right]}
%\newcommand{\LRa}[1]{\left\langle #1 \right\rangle}
%\newcommand{\LRb}[1]{\left| #1 \right|}
%\newcommand{\LRc}[1]{\left\{ #1 \right\}}
%\newcommand{\LRceil}[1]{\left\lceil #1 \right\rceil}
\newcommand{\LRl}[1]{\left. \LRp{#1} \right|}
%\newcommand{\jump}[1] {\ensuremath{\llbracket#1\rrbracket}}
%\newcommand{\avg}[1] {\ensuremath{\LRc{\!\{#1\}\!}}}
%\newcommand{\Grad} {\ensuremath{\nabla}}
%\renewcommand{\d}{\partial}
\newcommand{\diag}[1]{{\rm diag}\LRp{#1}}

\newcommand{\note}[1]{{\color{blue}{#1}}}
\newcommand{\bnote}[1]{{\color{blue}{#1}}}
\newcommand{\rnote}[1]{{\color{red}{#1}}}


\newcommand{\LK}{L^2\LRp{D^k}}
\newcommand{\LdK}{L^2\LRp{\partial D^k}}
\newcommand{\Dhat}{\widehat{D}}
\newcommand{\Lhat}{L^2\LRp{\Dhat}}


\newcommand*\diff[1]{\mathop{}\!{\mathrm{d}#1}} % d in integrand

\begin{document}

%\maketitle

I am very grateful to both \bnote{Reviewer 1} and \rnote{Reviewer 2} for their feedback.  I describe steps taken to address reviewer comments and suggestions in the following response.  I have also edited a few additional sections of the manuscript to improve their readability, and have added some more recent references.
\\
\\
Revisions in the manuscript are also colored for ease of identification.  We hope these revisions improve the readability of this paper and its suitability for the audience of JSC.

\section{Reviewer 1}

\textit{My main concern with this paper is the discussion of what has been accomplished. Specifically, the paper reads as if the SBP property is not required; however, this is not the case and what the author has presented is a means of constructing decoupled SBP operators using nearly arbitrary surface and volume quadrature nodes. }

\bnote{I wholeheartedly agree with Reviewer 1's assessment of the structure of the paper, and thank them for bringing this to my attention.  I have restructured the paper to make this more clear, and have added references to make connections to similar approaches in the SBP literature.  The main restructuring shows how to construct the SBP operator, and utilizes the ``variational'' SBP property only to prove accuracy properties.  The paper is also shorter and more concise after these revisions.}

\begin{enumerate}
\item \textit{The major comment is that the discussion throughout the paper on the strong SBP property vs. the variational SBP property does not clearly transmit what the author has accomplished in this manuscript and is misleading. The author gives a systematic approach to the construction of modal decoupled SBP operators from nearly arbitrary sets of volume and surface quadrature rules. Note that the skew-symmetric operators that are used in the algorithm satisfy the strong SBP property. What the author needs to clearly state is that the construction of decoupled SBP operators proposed in (Chan 2018) can lose their SBP property for specific choices of volume and surface quadrature rules and that this paper shows how to remedy this issue.}. 

\textit{It would be helpful [\ldots] to point out that the presented formulation is equivalent to using the newly defined decoupled SBP operator in computational space as applied to a skew-symmetric splitting of the conservation law.}

\bnote{I have rewritten significant portions of the manuscript.  The reviewer's observation that the skew-symmetric operators constructed satisfy the strong SBP property significantly simplifies proofs of the key lemmas and theorems.  The revised manuscript simply utilizes the strong SBP property now, and shifts the theoretical work in proving entropy conservation to showing that the reference skew-symmetric operators satisfy $\tilde{\bm{Q}}^i_N\bm{1} = \bm{0}$, and that the physical SBP operators in computational space satisfy an analogous condition.}

\end{enumerate}

\section{Reviewer 2}

\textit{The author generalizes his approach from [6,7] to handle cases where the discrete operator does not satisfy the decoupled SBP property.  The paper was nicely written and enjoyable to read.  I can find few faults with it.  The one suggestion I have is that the author bring up the constraints on accuracy sooner (such as in the abstract).  Only toward the end do we find out that, well, you probably still want the decoupled SBP property, or degree $2N-1$ volume and degree $2N$ surface quadratures, since accuracy will suffer otherwise.}

\rnote{I agree, and have added this information to both the introduction and conclusions.  The minor comments and typos have also been addressed; however, several no longer apply to the revised manuscript after the significant revisions made in light of Reviewer 1's comments.  }

\begin{itemize}
\item \rnote{I have corrected the following errors pointed out by Reviewer 2:}
\begin{itemize}
\item Page 2, Lines 16-19: I think Tadmor's 1987 work should be cited here too, no?
\item Page 3, Eq (1): subscript h on u, and should the zero on the RHS be bold?
\item Page 5, lines 5-6: The subscript on the basis does not match the index on the set.
\item Page 11, line 18: Here and elsewhere (5.3) seems to be a broken reference to (10).
\item Page 12, Eq (12): Ordinary derivative for the temporal derivative?
\item Page 15, line 43: As with (5.3), (6.1) seems to be a broken reference.
\item Page 16, line 47: Bold zero in GCL?
\item Page 19, line 34: "...surface quadrature on for which ..." typo  
\item Page 20, lines 7-8: "...satisfy Assumption 1 holds for v = 1."  typo
\item Page 26, line 13:  Search for "on on"
\item Page 26, line 50: "...we fix the volume to quadrature..."  Delete "to" ?
\end{itemize}
\item \textit{Page 12, Eq (14): I think $\psi_i$ should probably be a vector here for consistency.}

\rnote{I believe $\psi$ should remain a scalar here, since it's a function of a vector quantity.  To clarify this, however, I have added an extra line: ``Here, $\psi_i(\tilde{\bm{u}}_f)$ denotes the function $\psi_i$ evaluated at the face values of the entropy-projected conservative variables $\tilde{\bm{u}}_f$''.}

\item \textit{Page 15, line 21: Some ideas here go back to Thomas and Lombard, "Geometric conservation law and its application to flow computations on moving grids."}

\rnote{I have added this reference to the revised manuscript.}
\end{itemize}

\bibliographystyle{unsrt}
\bibliography{dg}

\end{document}

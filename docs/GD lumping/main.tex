\pdfoutput=1

%\documentclass[preprint,10pt]{elsarticle}
\documentclass[preprint,10pt]{amsart}
%\documentclass[review]{siamart0216}
%\documentclass{siamart0216}

%\usepackage{fullpage}
%\usepackage[colorlinks=true]{hyperref}

\usepackage{amsmath,amssymb,amsfonts}
%\usepackage{amsthm}
%\theoremstyle{definition}
%\newtheorem{definition}{Definition}
\theoremstyle{lemma}
\newtheorem{lemma}{Lemma}
\theoremstyle{corollary}
\newtheorem{corollary}{Corollary}
\newtheorem*{remark}{Remark}
\theoremstyle{theorem}
\newtheorem{theorem}{Theorem}
%\theoremstyle{assumption}
%\newtheorem{assumption}{Assumption}

\usepackage[titletoc,toc,title]{appendix}

\usepackage{array} 
\usepackage{listings}
\usepackage{mathtools}
\usepackage{pdfpages}
\usepackage[textsize=footnotesize,color=green]{todonotes}
\usepackage{bm}
\usepackage{bbm}

\usepackage{tikz}
\usepackage[normalem]{ulem}
\usepackage{hhline}

\usepackage{graphicx}
\usepackage{subfig}
\usepackage{color}

%% ====================================== graphics

\usepackage{pgfplots}
\usepackage{pgfplotstable}
\definecolor{markercolor}{RGB}{124.9, 255, 160.65}
\pgfplotsset{
compat=1.3,
width=10cm,
tick label style={font=\small},
label style={font=\small},
legend style={font=\small}
}

\usetikzlibrary{calc}
\usetikzlibrary{intersections} 

%%% START MACRO FOR ANNOTATION OF TRIANGLE WITH SLOPE %%%.
\newcommand{\logLogSlopeTriangle}[5]
{
    % #1. Relative offset in x direction.
    % #2. Width in x direction, so xA-xB.
    % #3. Relative offset in y direction.
    % #4. Slope d(y)/d(log10(x)).
    % #5. Plot options.

    \pgfplotsextra
    {
        \pgfkeysgetvalue{/pgfplots/xmin}{\xmin}
        \pgfkeysgetvalue{/pgfplots/xmax}{\xmax}
        \pgfkeysgetvalue{/pgfplots/ymin}{\ymin}
        \pgfkeysgetvalue{/pgfplots/ymax}{\ymax}

        % Calculate auxilliary quantities, in relative sense.
        \pgfmathsetmacro{\xArel}{#1}
        \pgfmathsetmacro{\yArel}{#3}
        \pgfmathsetmacro{\xBrel}{#1-#2}
        \pgfmathsetmacro{\yBrel}{\yArel}
        \pgfmathsetmacro{\xCrel}{\xArel}

        \pgfmathsetmacro{\lnxB}{\xmin*(1-(#1-#2))+\xmax*(#1-#2)} % in [xmin,xmax].
        \pgfmathsetmacro{\lnxA}{\xmin*(1-#1)+\xmax*#1} % in [xmin,xmax].
        \pgfmathsetmacro{\lnyA}{\ymin*(1-#3)+\ymax*#3} % in [ymin,ymax].
        \pgfmathsetmacro{\lnyC}{\lnyA+#4*(\lnxA-\lnxB)}
        \pgfmathsetmacro{\yCrel}{\lnyC-\ymin)/(\ymax-\ymin)} % THE IMPROVED EXPRESSION WITHOUT 'DIMENSION TOO LARGE' ERROR.

        % Define coordinates for \draw. MIND THE 'rel axis cs' as opposed to the 'axis cs'.
        \coordinate (A) at (rel axis cs:\xArel,\yArel);
        \coordinate (B) at (rel axis cs:\xBrel,\yBrel);
        \coordinate (C) at (rel axis cs:\xCrel,\yCrel);

        % Draw slope triangle.
        \draw[#5]   (A)-- node[pos=0.5,anchor=north] {}
                    (B)-- 
                    (C)-- node[pos=0.5,anchor=west] {\textcolor{black}{#4}}
                    cycle;
    }
}
%%% END MACRO FOR ANNOTATION OF TRIANGLE WITH SLOPE %%%.

\newcommand{\logLogSlopeTriangleNeg}[5]
{
    % #1. Relative offset in x direction.
    % #2. Width in x direction, so xA-xB.
    % #3. Relative offset in y direction.
    % #4. Slope d(y)/d(log10(x)).
    % #5. Plot options.

    \pgfplotsextra
    {
        \pgfkeysgetvalue{/pgfplots/xmin}{\xmin}
        \pgfkeysgetvalue{/pgfplots/xmax}{\xmax}
        \pgfkeysgetvalue{/pgfplots/ymin}{\ymin}
        \pgfkeysgetvalue{/pgfplots/ymax}{\ymax}

        % Calculate auxilliary quantities, in relative sense.
        \pgfmathsetmacro{\xArel}{#1}
        \pgfmathsetmacro{\yArel}{#3}
        \pgfmathsetmacro{\xBrel}{#1-#2}
        \pgfmathsetmacro{\yBrel}{\yArel}
        \pgfmathsetmacro{\xCrel}{\xArel}

        \pgfmathsetmacro{\lnxB}{\xmin*(1-(#1-#2))+\xmax*(#1-#2)} % in [xmin,xmax].
        \pgfmathsetmacro{\lnxA}{\xmin*(1-#1)+\xmax*#1} % in [xmin,xmax].
        \pgfmathsetmacro{\lnyA}{\ymin*(1-#3)+\ymax*#3} % in [ymin,ymax].
        \pgfmathsetmacro{\lnyC}{\lnyA+#4*(\lnxA-\lnxB)}
        \pgfmathsetmacro{\yCrel}{\lnyC-\ymin)/(\ymax-\ymin)} % THE IMPROVED EXPRESSION WITHOUT 'DIMENSION TOO LARGE' ERROR.

        % Define coordinates for \draw. MIND THE 'rel axis cs' as opposed to the 'axis cs'.
        \coordinate (A) at (rel axis cs:\xArel,\yArel);
        \coordinate (B) at (rel axis cs:\xBrel,\yBrel);
        \coordinate (C) at (rel axis cs:\xCrel,\yCrel);

        % Draw slope triangle.
        \draw[#5]   (A)-- node[pos=.5,anchor=south] {}
                    (B)-- 
                    (C)-- node[pos=0.5,anchor=west] {\textcolor{black}{#4}}
                    cycle;
    }
}
%%% END MACRO FOR ANNOTATION OF TRIANGLE WITH SLOPE %%%.

%%% START MACRO FOR ANNOTATION OF TRIANGLE WITH SLOPE %%%.
\newcommand{\logLogSlopeTriangleFlipNeg}[5]
{
    % #1. Relative offset in x direction.
    % #2. Width in x direction, so xA-xB.
    % #3. Relative offset in y direction.
    % #4. Slope d(y)/d(log10(x)).
    % #5. Plot options.

    \pgfplotsextra
    {
        \pgfkeysgetvalue{/pgfplots/xmin}{\xmin}
        \pgfkeysgetvalue{/pgfplots/xmax}{\xmax}
        \pgfkeysgetvalue{/pgfplots/ymin}{\ymin}
        \pgfkeysgetvalue{/pgfplots/ymax}{\ymax}

        % Calculate auxilliary quantities, in relative sense.
        %\pgfmathsetmacro{\xArel}{#1}
        %\pgfmathsetmacro{\yArel}{#3}
        \pgfmathsetmacro{\xBrel}{#1-#2}
        \pgfmathsetmacro{\yBrel}{#3}
        \pgfmathsetmacro{\xCrel}{#1}

        \pgfmathsetmacro{\lnxB}{\xmin*(1-(#1-#2))+\xmax*(#1-#2)} % in [xmin,xmax].
        \pgfmathsetmacro{\lnxA}{\xmin*(1-#1)+\xmax*#1} % in [xmin,xmax].
        \pgfmathsetmacro{\lnyA}{\ymin*(1-#3)+\ymax*#3} % in [ymin,ymax].
        \pgfmathsetmacro{\lnyC}{\lnyA+#4*(\lnxA-\lnxB)}
        \pgfmathsetmacro{\yCrel}{\lnyC-\ymin)/(\ymax-\ymin)} % THE IMPROVED EXPRESSION WITHOUT 'DIMENSION TOO LARGE' ERROR.

	\pgfmathsetmacro{\xArel}{\xBrel}
        \pgfmathsetmacro{\yArel}{\yCrel}

        % Define coordinates for \draw. MIND THE 'rel axis cs' as opposed to the 'axis cs'.
        \coordinate (A) at (rel axis cs:\xArel,\yArel);
        \coordinate (B) at (rel axis cs:\xBrel,\yBrel);
        \coordinate (C) at (rel axis cs:\xCrel,\yCrel);

        % Draw slope triangle.
        \draw[#5]   (A)-- node[pos=0.5,anchor=east] {\textcolor{black}{#4}}
                    (B)-- 
                    (C)-- node[pos=0.5,anchor=north] {1}
                    cycle;
    }
}
%%% END MACRO FOR ANNOTATION OF TRIANGLE WITH SLOPE %%%.


%%% START MACRO FOR ANNOTATION OF TRIANGLE WITH SLOPE %%%.
\newcommand{\logLogSlopeTriangleFlip}[5]
{
    % #1. Relative offset in x direction.
    % #2. Width in x direction, so xA-xB.
    % #3. Relative offset in y direction.
    % #4. Slope d(y)/d(log10(x)).
    % #5. Plot options.

    \pgfplotsextra
    {
        \pgfkeysgetvalue{/pgfplots/xmin}{\xmin}
        \pgfkeysgetvalue{/pgfplots/xmax}{\xmax}
        \pgfkeysgetvalue{/pgfplots/ymin}{\ymin}
        \pgfkeysgetvalue{/pgfplots/ymax}{\ymax}

        % Calculate auxilliary quantities, in relative sense.
        %\pgfmathsetmacro{\xArel}{#1}
        %\pgfmathsetmacro{\yArel}{#3}
        \pgfmathsetmacro{\xBrel}{#1-#2}
        \pgfmathsetmacro{\yBrel}{#3}
        \pgfmathsetmacro{\xCrel}{#1}

        \pgfmathsetmacro{\lnxB}{\xmin*(1-(#1-#2))+\xmax*(#1-#2)} % in [xmin,xmax].
        \pgfmathsetmacro{\lnxA}{\xmin*(1-#1)+\xmax*#1} % in [xmin,xmax].
        \pgfmathsetmacro{\lnyA}{\ymin*(1-#3)+\ymax*#3} % in [ymin,ymax].
        \pgfmathsetmacro{\lnyC}{\lnyA+#4*(\lnxA-\lnxB)}
        \pgfmathsetmacro{\yCrel}{\lnyC-\ymin)/(\ymax-\ymin)} % THE IMPROVED EXPRESSION WITHOUT 'DIMENSION TOO LARGE' ERROR.

	\pgfmathsetmacro{\xArel}{\xBrel}
        \pgfmathsetmacro{\yArel}{\yCrel}

        % Define coordinates for \draw. MIND THE 'rel axis cs' as opposed to the 'axis cs'.
        \coordinate (A) at (rel axis cs:\xArel,\yArel);
        \coordinate (B) at (rel axis cs:\xBrel,\yBrel);
        \coordinate (C) at (rel axis cs:\xCrel,\yCrel);

        % Draw slope triangle.
        \draw[#5]   (A)-- node[pos=0.5,anchor=east] {\textcolor{black}{#4}}
                    (B)-- 
                    (C)-- node[pos=0.5,anchor=south] {}
                    cycle;
    }
}
%%% END MACRO FOR ANNOTATION OF TRIANGLE WITH SLOPE %%%.



\usepackage{stmaryrd}

\renewcommand{\tilde}{\widetilde}
\renewcommand{\hat}{\widehat}
\renewcommand{\topfraction}{0.85}
\renewcommand{\textfraction}{0.1}
\renewcommand{\floatpagefraction}{0.75}

\newcommand{\vect}[1]{\ensuremath\boldsymbol{#1}}
\newcommand{\tensor}[1]{\underline{\bm{#1}}}
\newcommand{\del}{\triangle}
\newcommand{\curl}{\grad \times}
\renewcommand{\div}{\grad \cdot}

\newcommand{\bbm}[1]{\mathbbm{#1}}
\newcommand{\bs}[1]{\boldsymbol{#1}}
\newcommand{\equaldef}{\stackrel{\mathrm{def}}{=}}

\newcommand{\td}[2]{\frac{{\rm d}#1}{{\rm d}{\rm #2}}}
\newcommand{\pd}[2]{\frac{\partial#1}{\partial#2}}
\newcommand{\nor}[1]{\left\| #1 \right\|}
\newcommand{\LRp}[1]{\left( #1 \right)}
\newcommand{\LRs}[1]{\left[ #1 \right]}
\newcommand{\LRa}[1]{\left\langle #1 \right\rangle}
\newcommand{\LRb}[1]{\left| #1 \right|}
\newcommand{\LRc}[1]{\left\{ #1 \right\}}
\newcommand{\LRceil}[1]{\left\lceil #1 \right\rceil}
\newcommand{\LRl}[1]{\left. #1 \right|}
\newcommand{\pdd}[2]{\frac{\partial^2#1}{\partial#2^2}}
\newcommand{\pdn}[3]{\frac{\partial^{#3}#1}{\partial#2^{#3}}}
\newcommand{\mb}[1]{\mathbf{#1}}
\newcommand{\mbb}[1]{\mathbb{#1}}
\newcommand{\mc}[1]{\mathcal{#1}}
\newcommand{\snor}[1]{\left| #1 \right|}


%\newcommand{\cond}[1]{\kappa\LRp{#1}}
\newcommand{\cond}[2]{\nor{#1}_{#2}\nor{{#1}^{-1}}_{#2}}


\newcommand{\Grad} {\ensuremath{\nabla}}
\newcommand{\Div} {\ensuremath{\nabla\cdot}}
\newcommand{\jump}[1] {\ensuremath{\llbracket#1\rrbracket}}
\newcommand{\avg}[1] {\ensuremath{\LRc{\!\{#1\}\!}}}

\newcommand{\Oh}{{\Omega_h}}
\renewcommand{\L}{L^2\LRp{\Omega}}
\newcommand{\LK}{L^2\LRp{D^k}}
\newcommand{\LdK}{L^2\LRp{\partial D^k}}
\newcommand{\Dhat}{\widehat{D}}
\newcommand{\Lhat}{L^2\LRp{\Dhat}}

\newcommand{\eval}[2][\right]{\relax
  \ifx#1\right\relax \left.\fi#2#1\rvert}

\def\etal{{\it et al.~}}


\newcommand{\note}[1]{{\color{blue}{#1}}}
%\newcommand{\noteOne}[1]{{\color{blue}{#1}}}
%\newcommand{\noteTwo}[1]{{\color{red}{#1}}}
%\newcommand{\note}[1]{#1}
%\newcommand{\noteOne}[1]{#1}
%\newcommand{\noteTwo}[1]{#1}


\newcommand{\LinfDk}{L^{\infty}\LRp{D^k}}

\newcommand{\diag}[1]{{\rm diag}\LRp{#1}}

\newcommand{\Ksub}{K_{\rm sub}}

\newcolumntype{C}[1]{>{\centering\let\newline\\\arraybackslash\hspace{0pt}}m{#1}}

%% d in integrand
\newcommand*\diff[1]{\mathop{}\!{\mathrm{d}#1}}

\makeatletter
\renewcommand\d[1]{\mspace{6mu}\mathrm{d}#1\@ifnextchar\d{\mspace{-3mu}}{}}
\makeatother

%\date{}
\author{}
\title{}
\graphicspath{{./figs/}}

\begin{document}

\section{Properties of GD functions}

Interior GD functions are defined as scalings and translations of a ``canonical'' GD function $\phi(x)$.  Assuming that the polynomial degree $p$ is odd, let $\ell_j(x)$ be the $(p+1)$ Lagrange functions defined using equispaced nodal points $x_j$ such that
\[
\LRc{x_0,\ldots, x_p} = \LRc{-(a-1), \ldots, a}, \qquad a = \frac{p+1}{2}.%\LRceil{\frac{p}{2}}.
\]
This choice of nodal points ensures that the interval $[0,1]$ lies is the middle of $[x_0, x_p]$.  We define the canonical GD function $\phi(x)$ piecewise on the intervals $I_0,\ldots, I_{2p-2}$, where $I_j = [-(p-1)+j,-p+j]$
\[
\phi(x) = \begin{cases}
\ell_j(x-j), & x \in I_j\\
0 & {\rm otherwise}
\end{cases}.
\]

Symbolic computations then yield the following lemma:
\begin{lemma}
Let $\phi$ be the canonical GD function.  For $p = 1,3,\ldots, 75$, 
\begin{enumerate}
\item $\int_{-\infty}^{\infty} \phi^2 \leq \int_{-\infty}^{\infty} \phi = 1$  \label{eq:lem1}
\item The GD basis functions satisfy \label{eq:lem2}
\[
\sum_{j=-p}^{p} \LRb{ \int \phi_0(x) \phi_0(x-j)} \leq 2.
\]
\item For any degree $p$ polynomial $u(x)$, \label{eq:lem3}
\[
\int_{-\infty}^{\infty} u\phi_i = u(0).
\]
\end{enumerate}
\label{lemma:gdprops}
\end{lemma}
\begin{proof}
We use symbolic software to verify the first part (\ref{eq:lem1}) and (\ref{eq:lem2}).  
The equality in (\ref{eq:lem1}) can be shown directly.  Because ${\rm supp}\LRp{\phi} = [-(p-1),p-1]$, $\int_{-\infty}^\infty \phi(x) = \int_{-(p-1)}^{p-1} \phi(x)$.  By the definition of $\phi(x)$, 
\[
\int_{-(p-1)}^{(p-1)} \phi(x) = \sum_{j=0}^p \int_{I_j} \ell_j(x-x_j) = \int_0^1 \sum_{j=0}^p \ell_j(x) = \int_0^1 1 = 1.
\]
To prove (\ref{eq:lem3}), we first show that $\int_{-(p-1)}^{p-1} x^k\phi = 0$ for $0 < k \leq p$.  For $k > 0$ odd, this holds by the symmetry of $\phi(x)$ across $0$.  For $0 < k \leq p$ even, this condition is equivalent to
\[
\int_{-(p-1)}^{p-1} x^k \phi_i(x) = 2 \int_{0}^{p-1} x^k \phi_i(x)= \sum_{j=0}^a \int_0^1 \ell_j(x) (x-j)^k = 0.
\]
which we verify for $p = 1,\ldots, 75$ using symbolic software.  
Then, (\ref{eq:lem3}) follows from (\ref{eq:lem1}) and a Taylor representation of $u(x)$ around $x = 0$.
\[
u(x) = u(0) + u'(0) x + u''(0) x^2 + \ldots + u^{(p)}(0) x^p.
\]
\end{proof}
We conjecture that Lemma~\ref{lemma:gdprops} holds for all $p > 0$ odd.  A translation and scaling of $\phi$ then provides the following corollary:
\begin{corollary}
Let $x_i$ be equispaced points with spacing $h$, and let $\phi_i(x) = \phi\LRp{(x - x_j)/h}$ be the GD function at $x_i$.  For any degree $p$ polynomial $u(x)$, 
\[
\frac{1}{h}\int_{-\infty}^{\infty} u\phi_i = u(x_i).
\]
\label{cor:acc}
\end{corollary}

\section{Accuracy and energy stability of the lumped GD mass matrix}

Using Lemma~\ref{lemma:gdprops}, we can show that the lumped GD mass matrix is high order accurate in the following sense:
\begin{lemma}
Let $\tilde{\bm{M}}$ denote the non-symmetric lumped GD mass matrix.  Then, $\tilde{\bm{M}}^{-1}\bm{Q}$ is a $(p+1)$ order accurate approximation to the first derivative.  
\end{lemma}
\begin{proof}
Let $u(x)$ be a degree $p$ polynomial with GD coefficients $\bm{u}_i = u(x_i)$.  Since the GD basis can reproduce polynomials of degree $p$, $\bm{\delta u} = \bm{M}^{-1}\bm{Q}\bm{u}$ are the GD coefficients of the exact derivative $\LRl{\pd{u}{x}}_{x_i}$.  

The lumped GD mass matrix $\tilde{\bm{M}}$ is $(p+1)$ order accurate if $\tilde{\bm{M}} \bm{\delta u} = \bm{Q}\bm{u}$ as well.  Since $\bm{\delta u}$ is again polynomial, high order accuracy is ensured if $\bm{M}\bm{u} = \tilde{\bm{M}}\bm{u}$ for all coefficients $\bm{u}$ which correspond to polynomials of degree $p$.  Since the boundary rows of $\bm{M}$ and $\tilde{\bm{M}}$ are identical, $(\bm{M}\bm{u})_i = (\tilde{\bm{M}}\bm{u})_i$ for all indices $i$ corresponding to boundary GD functions.  For $\bm{u}$ polynomial and $i$ corresponding to interior GD functions, Corollary~\ref{cor:acc} guarantees $(\bm{M}\bm{u})_i = (\tilde{\bm{M}}\bm{u})_i$.  
\end{proof}

We now show that, for linear symmetric hyperbolic PDEs, the lumped GD mass matrix preserves semi-discrete energy stability.  We assume that there are sufficiently many elements relative to the order $p$.  Then, under a reordering of degrees of freedom, the GD mass matrix $\bm{M}$ is
\[
\bm{M} = \begin{bmatrix}
\bm{A} & \bm{B}\\
\bm{B}^T & \bm{C}
\end{bmatrix},
\]
where $\bm{A}$ is the sub-matrix consisting of rows and columns of $\bm{M}$ corresponding to boundary GD functions, and $\bm{C}$ is the sub-matrix corresponding to interior GD functions.  Since $\bm{M}$ is SPD, the matrices $\bm{A}, \bm{C}$ are also SPD.  We assume for simplicity that $h = 1$, such that the lumped mass matrix $\tilde{\bm{M}}$ is 
\[
{\tilde{\bm{M}}} = \begin{bmatrix}
\bm{A} & \bm{B}\\
\bm{0} & \bm{I}
\end{bmatrix}.
\]
We require that $\bm{x}^T\tilde{\bm{M}}\bm{x}$ induces a norm on $\bm{x}$, which holds if $\tilde{\bm{M}}$ is positive definite.  

%We can prove positive-definiteness using 
%\begin{lemma}
%Let $\bm{A}$ be a symmetric banded Toeplitz matrix
%\[
%\begin{bmatrix}
%a_0 & a_1 & \ldots & a_p &&&\\
%a_1 & a_0 & a_1 & \ldots &a_p &&\\
%\vdots & \ddots & \ddots & \ddots& &&\\
%\end{bmatrix}
%\]
%The spectral radius of $\bm{A}$ is bounded by $\sum_{j=0}^p \LRb{a_j}$.
%\label{lemma:banded}
%\end{lemma}

For $h = 1$, we have the following property of $\bm{C}$:
\begin{lemma}
The maximum eigenvalue of $\LRb{\bm{C}}$ is bounded by $2$.
\label{lemma:Ceig}
\end{lemma}
\begin{proof}
The proof follows directly from bounding the Rayleigh quotient of $\LRb{\bm{C}}$ and using the banded Toeplitz nature of the matrix.  Expanding $\bm{u}^T\LRb{\bm{C}}\bm{u}$ out and using symmetry gives
\begin{align*}
\bm{u}^T\LRb{\bm{C}}\bm{u} &\leq \sum_{i=0}^K\sum_{j = \max(0,i-p)}^{\min(K,i+p)} \LRb{\bm{u}_i\bm{u}_j}\LRb{\int \phi_i\phi_j}\\
&\leq \sum_{i=0}^K \LRp{\bm{u}_i^2 \int\phi_i^2 + 2\sum_{j=i+1}^{\min(K,i+p)} \LRb{\bm{u}_i\bm{u}_j} \LRb{\int\phi_i\phi_j}}.
\end{align*}
Applying Young's inequality bounds this sum from above by
\begin{align}
&\leq\sum_{i=0}^K \LRp{\bm{u}_i^2 \int\phi_i^2 + \sum_{j=i+1}^{\min(K,i+p)} \LRp{\bm{u}_i^2 + \bm{u}_j^2} \LRb{\int\phi_i\phi_j}} \nonumber\\
&= \sum_{i=0}^K \bm{u}_i^2 \LRp{\sum_{j=i}^{\min(K,i+p)} \LRb{\int\phi_i\phi_{j}}} + \sum_{j=i+1}^{i+p} \bm{u}_j^2\LRb{\int \phi_i\phi_j}.\label{eq:tmp1}
\end{align}
Distributing the terms of the latter sum in (\ref{eq:tmp1}) yields that
\begin{align*}
\bm{u}^T\LRb{\bm{C}}\bm{u} &\leq \sum_{i=0}^K \bm{u}_i^2 \LRp{\sum_{j=\max(i-p,0)}^{\min(K,i+p)} \LRb{\int\phi_i\phi_{j}}}\\
&\leq  \sum_{i=0}^K \bm{u}_i^2\LRp{\sum_{j=-p}^{p} \LRb{\int\phi_0\phi_j}} \leq 2\sum_{i=0}^K \bm{u}_i^2  = 2\bm{u}^T\bm{u}
\end{align*}
where we have used translation invariance of the interior GD basis functions and property (\ref{eq:lem2}) of Lemma~\ref{lemma:gdprops}.
%\[
%\LRb{\int \phi_i\phi_{i+j}} = \LRb{\int \phi_0\phi_{j}}, \qquad -p \leq j \leq p.
%\]
%, using yields
%\[
%\bm{u}^T\LRb{\bm{C}}\bm{u} \leq \sum_{i=0}^K \bm{u}_i^2 \LRp{\sum_{j=-p}^{p} \LRb{\int\phi_0\phi_j}} \leq 2\sum_{i=0}^K \bm{u}_i^2  = 2\bm{u}^T\bm{u}.
%\]
\end{proof}

We can then show the following:
\begin{lemma}
The lumped GD mass matrix $\tilde{\bm{M}}$ is positive definite in the sense that
\[
0 < \bm{x}^T\tilde{\bm{M}}\bm{x}, \qquad \forall \bm{x}\in \mathbb{R}^n, \quad \bm{x}\neq 0.
\]
\end{lemma}
\begin{proof}
Let $\bm{x} = \LRs{\bm{u},\bm{v}}^T$.  Using that $\bm{u}^T\bm{C}\bm{u} \leq \bm{u}^T\LRb{\bm{C}}\bm{u}$ and Lemma~\ref{lemma:Ceig}
\begin{align*}
0 < \bm{x}^T{\bm{M}}\bm{x}&\leq \bm{u}^T\bm{A}\bm{u} + 2\bm{u}^T\bm{B}\bm{v} + \bm{v}^T\LRb{\bm{C}}\bm{v}\\
&\leq \bm{u}^T\bm{A}\bm{u} + 2\bm{u}^T\bm{B}\bm{v} + 2\bm{v}^T\bm{v}\\
&\leq 2\LRp{\bm{u}^T\bm{A}\bm{u} + \bm{u}^T\bm{B}\bm{v}+ \bm{v}^T\bm{v}} = 2 \bm{x}^T\tilde{\bm{M}}\bm{x}
\end{align*}
\end{proof}

\begin{remark}
The bound in Lemma~\ref{lemma:Ceig} is sufficient to show positive definiteness of the non-symmetric lumped GD mass matrix.  However, numerical experiments indicate that the maximum eigenvalue $\lambda_{\max}$ of $\bm{C}$ achieves a tighter bound $\lambda_{\max} \leq 1$ for all $p$ and $K$ tested.  
\end{remark}


Ideally, the norms induced by $\bm{M}$ and $\tilde{\bm{M}}$ should also be equivalent and induce equivalent measures of energy.  Numerical experiments that this is indeed the case.  

%\section{Matlab code for symbolic computations}
%
%\begin{verbatim}
%clear
%syms x
%for p = 1:2:25
%    a = (p+1)/2;   
%    for k = 2:2:p-1
%        moment = sym(0);
%        phi2 = sym(0);
%        for j = -(a-1):0 % half of the points
%            ellj = sym(1);
%            denom = sym(1);
%            for i = -(a-1):a
%                if i~=j
%                    ellj = ellj * (x-i);
%                    denom = denom * (j-i);
%                end
%            end
%            ellj = ellj/denom;
%            moment = moment + int(ellj*(x-j)^k,0,1);
%            phi2 = phi2 + int(ellj*ellj,0,1);
%        end
%        if ~isequaln(moment,sym(0)) || (phi2 > 1.0)
%            fprintf('Condition violated for p = %d\n',p)
%            return
%        end
%    end
%    fprintf('Done with p = %d\n',p)
%end
%\end{verbatim}

\end{document}



\pdfoutput=1

%\documentclass[preprint,10pt]{elsarticle}
\documentclass[preprint,10pt]{article}
%\documentclass[review]{siamart0216}
%\documentclass{siamart0216}

\usepackage{fullpage}
\usepackage{amsmath,amssymb,amsfonts,amsthm}
\theoremstyle{definition}
\newtheorem{definition}{Definition}[section]
\newtheorem*{remark}{Remark}
%\usepackage{thmtools}
%\declaretheorem[style=definition,qed=$\blacksquare$,numberwithin=chapter]{definition}

\usepackage[titletoc,toc,title]{appendix}

\usepackage{array} 
\usepackage{listings}
\usepackage{mathtools}
\usepackage{pdfpages}
\usepackage[textsize=footnotesize,color=green]{todonotes}
\usepackage{bm}
\usepackage{bbm}

%\usepackage{tikz}
\usepackage[normalem]{ulem}
\usepackage{hhline}

%% ====================================== alg package
\usepackage{algorithm}
\usepackage[noend]{algpseudocode}
\usepackage{algorithmicx}
\algblock{ParFor}{EndParFor}
% customising the new block
\algnewcommand\algorithmicparfor{\textbf{parfor}}
\algnewcommand\algorithmicpardo{\textbf{do}}
\algnewcommand\algorithmicendparfor{\textbf{end\ parfor}}
\algrenewtext{ParFor}[1]{\algorithmicparfor\ #1\ \algorithmicpardo}
\algrenewtext{EndParFor}{\algorithmicendparfor}
%% ====================================== end alg package

\usepackage{graphicx}
\usepackage{subfig}
\usepackage{color}

%% ====================================== graphics

\usepackage{pgfplots}
\usepackage{pgfplotstable}
\definecolor{markercolor}{RGB}{124.9, 255, 160.65}
\pgfplotsset{width=10cm,compat=1.3}
\pgfplotsset{
tick label style={font=\small},
label style={font=\small},
legend style={font=\small}
}

\usetikzlibrary{calc}

%%% START MACRO FOR ANNOTATION OF TRIANGLE WITH SLOPE %%%.
\newcommand{\logLogSlopeTriangle}[5]
{
    % #1. Relative offset in x direction.
    % #2. Width in x direction, so xA-xB.
    % #3. Relative offset in y direction.
    % #4. Slope d(y)/d(log10(x)).
    % #5. Plot options.

    \pgfplotsextra
    {
        \pgfkeysgetvalue{/pgfplots/xmin}{\xmin}
        \pgfkeysgetvalue{/pgfplots/xmax}{\xmax}
        \pgfkeysgetvalue{/pgfplots/ymin}{\ymin}
        \pgfkeysgetvalue{/pgfplots/ymax}{\ymax}

        % Calculate auxilliary quantities, in relative sense.
        \pgfmathsetmacro{\xArel}{#1}
        \pgfmathsetmacro{\yArel}{#3}
        \pgfmathsetmacro{\xBrel}{#1-#2}
        \pgfmathsetmacro{\yBrel}{\yArel}
        \pgfmathsetmacro{\xCrel}{\xArel}

        \pgfmathsetmacro{\lnxB}{\xmin*(1-(#1-#2))+\xmax*(#1-#2)} % in [xmin,xmax].
        \pgfmathsetmacro{\lnxA}{\xmin*(1-#1)+\xmax*#1} % in [xmin,xmax].
        \pgfmathsetmacro{\lnyA}{\ymin*(1-#3)+\ymax*#3} % in [ymin,ymax].
        \pgfmathsetmacro{\lnyC}{\lnyA+#4*(\lnxA-\lnxB)}
        \pgfmathsetmacro{\yCrel}{\lnyC-\ymin)/(\ymax-\ymin)} % THE IMPROVED EXPRESSION WITHOUT 'DIMENSION TOO LARGE' ERROR.

        % Define coordinates for \draw. MIND THE 'rel axis cs' as opposed to the 'axis cs'.
        \coordinate (A) at (rel axis cs:\xArel,\yArel);
        \coordinate (B) at (rel axis cs:\xBrel,\yBrel);
        \coordinate (C) at (rel axis cs:\xCrel,\yCrel);

        % Draw slope triangle.
        \draw[#5]   (A)-- node[pos=0.5,anchor=north] {1}
                    (B)-- 
                    (C)-- node[pos=0.5,anchor=west] {#4}
                    cycle;
    }
}
%%% END MACRO FOR ANNOTATION OF TRIANGLE WITH SLOPE %%%.

\newcommand{\logLogSlopeTriangleNeg}[5]
{
    % #1. Relative offset in x direction.
    % #2. Width in x direction, so xA-xB.
    % #3. Relative offset in y direction.
    % #4. Slope d(y)/d(log10(x)).
    % #5. Plot options.

    \pgfplotsextra
    {
        \pgfkeysgetvalue{/pgfplots/xmin}{\xmin}
        \pgfkeysgetvalue{/pgfplots/xmax}{\xmax}
        \pgfkeysgetvalue{/pgfplots/ymin}{\ymin}
        \pgfkeysgetvalue{/pgfplots/ymax}{\ymax}

        % Calculate auxilliary quantities, in relative sense.
        \pgfmathsetmacro{\xArel}{#1}
        \pgfmathsetmacro{\yArel}{#3}
        \pgfmathsetmacro{\xBrel}{#1-#2}
        \pgfmathsetmacro{\yBrel}{\yArel}
        \pgfmathsetmacro{\xCrel}{\xArel}

        \pgfmathsetmacro{\lnxB}{\xmin*(1-(#1-#2))+\xmax*(#1-#2)} % in [xmin,xmax].
        \pgfmathsetmacro{\lnxA}{\xmin*(1-#1)+\xmax*#1} % in [xmin,xmax].
        \pgfmathsetmacro{\lnyA}{\ymin*(1-#3)+\ymax*#3} % in [ymin,ymax].
        \pgfmathsetmacro{\lnyC}{\lnyA+#4*(\lnxA-\lnxB)}
        \pgfmathsetmacro{\yCrel}{\lnyC-\ymin)/(\ymax-\ymin)} % THE IMPROVED EXPRESSION WITHOUT 'DIMENSION TOO LARGE' ERROR.

        % Define coordinates for \draw. MIND THE 'rel axis cs' as opposed to the 'axis cs'.
        \coordinate (A) at (rel axis cs:\xArel,\yArel);
        \coordinate (B) at (rel axis cs:\xBrel,\yBrel);
        \coordinate (C) at (rel axis cs:\xCrel,\yCrel);

        % Draw slope triangle.
        \draw[#5]   (A)-- node[pos=.5,anchor=south] {1}
                    (B)-- 
                    (C)-- node[pos=0.5,anchor=west] {#4}
                    cycle;
    }
}
%%% END MACRO FOR ANNOTATION OF TRIANGLE WITH SLOPE %%%.

%%% START MACRO FOR ANNOTATION OF TRIANGLE WITH SLOPE %%%.
\newcommand{\logLogSlopeTriangleFlipNeg}[5]
{
    % #1. Relative offset in x direction.
    % #2. Width in x direction, so xA-xB.
    % #3. Relative offset in y direction.
    % #4. Slope d(y)/d(log10(x)).
    % #5. Plot options.

    \pgfplotsextra
    {
        \pgfkeysgetvalue{/pgfplots/xmin}{\xmin}
        \pgfkeysgetvalue{/pgfplots/xmax}{\xmax}
        \pgfkeysgetvalue{/pgfplots/ymin}{\ymin}
        \pgfkeysgetvalue{/pgfplots/ymax}{\ymax}

        % Calculate auxilliary quantities, in relative sense.
        %\pgfmathsetmacro{\xArel}{#1}
        %\pgfmathsetmacro{\yArel}{#3}
        \pgfmathsetmacro{\xBrel}{#1-#2}
        \pgfmathsetmacro{\yBrel}{#3}
        \pgfmathsetmacro{\xCrel}{#1}

        \pgfmathsetmacro{\lnxB}{\xmin*(1-(#1-#2))+\xmax*(#1-#2)} % in [xmin,xmax].
        \pgfmathsetmacro{\lnxA}{\xmin*(1-#1)+\xmax*#1} % in [xmin,xmax].
        \pgfmathsetmacro{\lnyA}{\ymin*(1-#3)+\ymax*#3} % in [ymin,ymax].
        \pgfmathsetmacro{\lnyC}{\lnyA+#4*(\lnxA-\lnxB)}
        \pgfmathsetmacro{\yCrel}{\lnyC-\ymin)/(\ymax-\ymin)} % THE IMPROVED EXPRESSION WITHOUT 'DIMENSION TOO LARGE' ERROR.

	\pgfmathsetmacro{\xArel}{\xBrel}
        \pgfmathsetmacro{\yArel}{\yCrel}

        % Define coordinates for \draw. MIND THE 'rel axis cs' as opposed to the 'axis cs'.
        \coordinate (A) at (rel axis cs:\xArel,\yArel);
        \coordinate (B) at (rel axis cs:\xBrel,\yBrel);
        \coordinate (C) at (rel axis cs:\xCrel,\yCrel);

        % Draw slope triangle.
        \draw[#5]   (A)-- node[pos=0.5,anchor=east] {#4}
                    (B)-- 
                    (C)-- node[pos=0.5,anchor=north] {1}
                    cycle;
    }
}
%%% END MACRO FOR ANNOTATION OF TRIANGLE WITH SLOPE %%%.


%%% START MACRO FOR ANNOTATION OF TRIANGLE WITH SLOPE %%%.
\newcommand{\logLogSlopeTriangleFlip}[5]
{
    % #1. Relative offset in x direction.
    % #2. Width in x direction, so xA-xB.
    % #3. Relative offset in y direction.
    % #4. Slope d(y)/d(log10(x)).
    % #5. Plot options.

    \pgfplotsextra
    {
        \pgfkeysgetvalue{/pgfplots/xmin}{\xmin}
        \pgfkeysgetvalue{/pgfplots/xmax}{\xmax}
        \pgfkeysgetvalue{/pgfplots/ymin}{\ymin}
        \pgfkeysgetvalue{/pgfplots/ymax}{\ymax}

        % Calculate auxilliary quantities, in relative sense.
        %\pgfmathsetmacro{\xArel}{#1}
        %\pgfmathsetmacro{\yArel}{#3}
        \pgfmathsetmacro{\xBrel}{#1-#2}
        \pgfmathsetmacro{\yBrel}{#3}
        \pgfmathsetmacro{\xCrel}{#1}

        \pgfmathsetmacro{\lnxB}{\xmin*(1-(#1-#2))+\xmax*(#1-#2)} % in [xmin,xmax].
        \pgfmathsetmacro{\lnxA}{\xmin*(1-#1)+\xmax*#1} % in [xmin,xmax].
        \pgfmathsetmacro{\lnyA}{\ymin*(1-#3)+\ymax*#3} % in [ymin,ymax].
        \pgfmathsetmacro{\lnyC}{\lnyA+#4*(\lnxA-\lnxB)}
        \pgfmathsetmacro{\yCrel}{\lnyC-\ymin)/(\ymax-\ymin)} % THE IMPROVED EXPRESSION WITHOUT 'DIMENSION TOO LARGE' ERROR.

	\pgfmathsetmacro{\xArel}{\xBrel}
        \pgfmathsetmacro{\yArel}{\yCrel}

        % Define coordinates for \draw. MIND THE 'rel axis cs' as opposed to the 'axis cs'.
        \coordinate (A) at (rel axis cs:\xArel,\yArel);
        \coordinate (B) at (rel axis cs:\xBrel,\yBrel);
        \coordinate (C) at (rel axis cs:\xCrel,\yCrel);

        % Draw slope triangle.
        \draw[#5]   (A)-- node[pos=0.5,anchor=east] {#4}
                    (B)-- 
                    (C)-- node[pos=0.5,anchor=south] {1}
                    cycle;
    }
}
%%% END MACRO FOR ANNOTATION OF TRIANGLE WITH SLOPE %%%.



\usepackage{stmaryrd}


\renewcommand{\topfraction}{0.85}
\renewcommand{\textfraction}{0.1}
\renewcommand{\floatpagefraction}{0.75}

\newcommand{\vect}[1]{\ensuremath\boldsymbol{#1}}
\newcommand{\tensor}[1]{\underline{\bm{#1}}}
\newcommand{\del}{\triangle}
\newcommand{\curl}{\grad \times}
\renewcommand{\div}{\grad \cdot}

\newcommand{\bbm}[1]{\mathbbm{#1}}
\newcommand{\bs}[1]{\boldsymbol{#1}}
\newcommand{\equaldef}{\stackrel{\mathrm{def}}{=}}

\newcommand{\td}[2]{\frac{{\rm d}#1}{{\rm d}{\rm #2}}}
\newcommand{\pd}[2]{\frac{\partial#1}{\partial#2}}
\newcommand{\pdd}[2]{\frac{\partial^2#1}{\partial#2^2}}
\newcommand{\pdn}[3]{\frac{\partial^{#3}#1}{\partial#2^{#3}}}
\newcommand{\mb}[1]{\mathbf{#1}}
\newcommand{\mbb}[1]{\mathbb{#1}}
\newcommand{\mc}[1]{\mathcal{#1}}
\newcommand{\snor}[1]{\left| #1 \right|}
\newcommand{\nor}[1]{\left\| #1 \right\|}
\newcommand{\LRp}[1]{\left( #1 \right)}
\newcommand{\LRs}[1]{\left[ #1 \right]}
\newcommand{\LRa}[1]{\left\langle #1 \right\rangle}
\newcommand{\LRb}[1]{\left| #1 \right|}
\newcommand{\LRc}[1]{\left\{ #1 \right\}}
\newcommand{\LRceil}[1]{\left\lceil #1 \right\rceil}
\newcommand{\LRl}[1]{\left. #1 \right|}

%\newcommand{\cond}[1]{\kappa\LRp{#1}}
\newcommand{\cond}[2]{\nor{#1}_{#2}\nor{{#1}^{-1}}_{#2}}


\newcommand{\Grad} {\ensuremath{\nabla}}
\newcommand{\Div} {\ensuremath{\nabla\cdot}}
\newcommand{\jump}[1] {\ensuremath{\llbracket#1\rrbracket}}
\newcommand{\avg}[1] {\ensuremath{\LRc{\!\{#1\}\!}}}

\newcommand{\Oh}{{\Omega_h}}
\renewcommand{\L}{L^2\LRp{\Omega}}
\newcommand{\LK}{L^2\LRp{D^k}}
\newcommand{\LdK}{L^2\LRp{\partial D^k}}
\newcommand{\Dhat}{\widehat{D}}
\newcommand{\Lhat}{L^2\LRp{\Dhat}}

\newcommand{\eval}[2][\right]{\relax
  \ifx#1\right\relax \left.\fi#2#1\rvert}

\def\etal{{\it et al.~}}


\newcommand{\note}[1]{{\color{blue}{#1}}}


\newcommand{\LinfDk}{L^{\infty}\LRp{D^k}}

\newcommand{\diag}[1]{{\rm diag}\LRp{#1}}

\newcommand{\Ksub}{K_{\rm sub}}

\newcolumntype{C}[1]{>{\centering\let\newline\\\arraybackslash\hspace{0pt}}m{#1}}

%% d in integrand
\newcommand*\diff[1]{\mathop{}\!{\mathrm{d}#1}}


\makeatletter
\renewcommand\d[1]{\mspace{6mu}\mathrm{d}#1\@ifnextchar\d{\mspace{-3mu}}{}}
\makeatother

\date{}
\author{Jesse Chan}
\title{Energy stable discontinuous Galerkin formulations and discrete differential operators}

\begin{document}

\maketitle

\begin{abstract}
We show that, for non-linear hyperbolic conservation laws which admit a skew-symmetric splitting, energy stable discontinuous Galerkin (DG) methods can be constructed in a straightforward manner based on discrete DG gradient and divergence operators.  In particular, we show that these formulations remain energy stable for both curvilinear geometries and inexact quadrature.  Examples of energy stable formulations are given for variable advection, Burgers' equation, and a Burgers'-like system, and it is shown that the construction of DG methods based on discrete differential operators recover known entropy-conservative fluxes.  
\end{abstract}

\section{Introduction}

\cite{feng2016discontinuous}

Lifting operators  and discrete gradients \cite{bassi1997high, di2011mathematical}

\section{Discrete differential operators}

We assume that the domain $\Omega$ is decomposed into non-overlapping elements $D^k$.  We define the mesh $\Omega_h = \cup D^k$ and the corresponding global approximation space $V_h(\Omega_h) = \bigoplus V_h\LRp{D^k}$, where $V_h\LRp{D^k}$ is the approximation space over a single patch.  Furthermore, we introduce the jump and average of discontinuous functions across element interfaces.  Let $D^{k,+}$ denote the neighboring element of across a face $f$ of $D^k$, and let $u^+,u^-$ denote the values of $u$ on $D^{k,+}$ and $D^k$, respectively.  The jump of $u$ across $f$ is then defined as
\[
\jump{u} = u^+ - u^-, \qquad \avg{u} = \frac{u^+ + u^-}{2}.
\]
The jump and average of vector fields are defined component-wise using the jumps and averages of components.  

\subsection{Discrete inner products}

In order to define discrete differential operators, we first introduce the $L^2$ inner product on $V_h\LRp{\Oh}$
\[
\LRp{u,v}_{\Omega} = \sum_{D^k} \int_{D^k} uv \diff{x} = \sum_{D^k} \int_{\widehat{D}} uv J \diff{\widehat{x}}.  
\]
In practice, these integrals are computed using quadrature, such that 
\[
\int_{\widehat{D}}uv = \sum_{i=1}^{N_q} u(\bm{x}_i)v(\bm{x}_i) w_i,
\]
where $N_q$ is the number of quadrature points.  The only assumptions we make upon this quadrature is that it is sufficiently accurate such that 
\begin{enumerate}
\item The quadrature induces an equivalent inner product over the reference element $\widehat{D}$
\item Integration by parts holds (with respect to the reference coordinates $\widehat{x}$).
\end{enumerate}

We also introduce the $L^2$ projection $\Pi_N: L^2\LRp{\Omega}\rightarrow P^N$ such that 
\[
\LRp{\Pi_N u,v}_{\Omega} = \LRp{u,v}_{\Omega}.
\]

\subsection{Discrete derivatives}
We introduce two discrete DG derivatives in this section.  The first is the implicit definition used in \cite{di2011mathematical}

\begin{definition}
The discontinuous Galerkin differentiation operator ${D}^i: L^2\LRp{\Omega}\rightarrow V_h$ is defined implicitly as follows:
\[
\LRp{{D}^i u,v}_{\Oh} = \sum_{D^k} \LRp{\LRp{-u,\pd{v}{\bm{x}_i}}_{\LK} + \LRa{\avg{u},v\bm{n}_i}_{\LdK}}.
\]
\end{definition}

We note that, for $u,v \in V_h$, integration by parts yields an equivalent definition of $D^i$ found in \cite{hesthaven2004high,Warburton20063205}
\[
\LRp{{D}^i u,v}_{\Oh} = \sum_{D^k} \LRp{\LRp{\pd{u}{\bm{x}_i},v}_{\LK} + \frac{1}{2}\LRa{\jump{u},v\bm{n}_i}_{\LdK}}.
\]
However, for $u \not\in V_h$, these two definitions are not equivalent.  This motivates the definition of a \textit{discrete} DG differentiation operator:
\begin{definition}
The discrete discontinuous Galerkin differentiation operator ${D}^i_h: L^2\LRp{\Omega} \rightarrow V_h$ is defined as $D^i_h = D^i \Pi_N$.
\end{definition}

The main property of 


\section{Notes}
Begin with the gradient.  Multiplication by $\bm{v}$ and integration by parts gives
\[
\sum_{D^k} \LRp{\Grad u, \bm{v}}_{\LK} = \sum_{D^k} \LRp{-u, \Grad \cdot \bm{v}}_{\LK} + \LRa{u,\bm{v}\cdot\bm{n}}_{\LdK}.
\]
We replace the values of ${u}$ with $\avg{u}$ on each element boundary to define the global DG gradient operator $\Grad_h$ 
\[
\sum_{D^k} \LRp{\Grad_h u, \bm{v}}_{\LK} \coloneqq \sum_{D^k} \LRp{-u, \Grad \cdot \bm{v}}_{\LK} + \LRa{\avg{u},\bm{v}\cdot\bm{n}}_{\LdK}.
\]
Integrating by parts again and the introduction of the lift operator shows that
\[
\Grad_h u = \Grad u + \frac{1}{2}L\LRp{\jump{u}\bm{n}}
\]
where $L$ is the lift operator.  

The DG divergence operator is similarly defined as
\[
\sum_{D^k} \LRp{\Grad_h \cdot \bm{u}, v}_{\LK} \coloneqq \sum_{D^k} \LRp{-\bm{u}, \Grad v}_{\LK} + \LRa{\avg{\bm{u}}\cdot\bm{n},v}_{\LdK}.
\]
and
\[
\Grad_h \cdot u = \Grad\cdot u + \frac{1}{2}L\LRp{\jump{\bm{u}}\cdot\bm{n}}
\]
and it can be shown that 
\[
\LRp{\Grad_h u, \bm{v}} = \LRp{-u, \Grad_h \cdot \bm{v}}.
\]
We will incorporate boundary conditions in a stable way in the following sections.  

When the support of $v$ is limited to a single element, we have
\[
\LRp{\Grad_h\cdot \bm{u},v\bbm{1}_{D^k}} = \LRp{\bm{u}, \Grad\cdot v\bbm{1}_{D^k}}  + \LRa{\avg{\bm{u}}\cdot\bm{n},v}_{\partial D^k}.
\]
and as a result when $v = 1$
\[
\LRp{\Grad_h\cdot \bm{u},\bbm{1}_{D^k}} = \int_{\partial D^k}\avg{\bm{u}}\cdot\bm{n}
\]
\section{Variable advection}

A split formulation for advection is 
\[
\LRp{\pd{u}{t},v} + \frac{1}{2}\LRp{\Grad_h\cdot \Pi_N \LRp{ \bm{\beta}u},v} + \frac{1}{2}\LRp{\bm{\beta}\cdot\Grad_h u,v} + \frac{1}{2}\LRp{\LRp{\Grad\cdot \bm{\beta}} u,v} = 0.
\]
Taking $v = u$ yields and using $\LRp{\Grad_h u, \bm{v}} = \LRp{-u, \Grad_h \cdot \bm{v}}$ yields the energy statement
\[
\frac{1}{2}\nor{u}^2 + \frac{1}{2}\LRp{\Grad_h\cdot \Pi_N \LRp{ \bm{\beta}u},u} - \frac{1}{2}\LRp{ u, \Grad_h\cdot \Pi_N\LRp{\bm{\beta}u}} = \frac{1}{2}\LRp{-\LRp{\Grad \cdot \bm{\beta}} u,u},
\]
implying that $\frac{1}{2}\nor{u}^2 = 0$ if $\Grad\cdot \bm{\beta} = 0$, or that the method is energy conserving.  The only difference in this formulation is the introduction of $\Pi_N$, which can be defined at a discrete level using any quadrature scheme for which a discrete projection is well-defined. 

\section{Local conservation}

Writing this in non-conservative form raises the question of local conservation.  Integrating the original equation over $D^k$ and using Gauss' theorem gives
\[
\int_{D^k}\pd{u}{t} + \int_{\partial D^k} \beta_n u = 0.
\]
Taking $v = 1$ on $D^k$ yields
\[
\int_{D^k}\pd{u}{t} + \frac{1}{2}\LRp{\Grad_h\cdot \Pi_N \LRp{ \bm{\beta}u},\bbm{1}_{D^k}} + \frac{1}{2}\LRp{\bm{\beta}\cdot\Grad_h u,\bbm{1}_{D^k}} + \frac{1}{2}\LRp{\LRp{\Grad\cdot \bm{\beta}} u,\bbm{1}_{D^k}} = 0.
\]
The first term gives
\[
\LRp{\Grad_h\cdot \Pi_N \LRp{ \bm{\beta}u},\bbm{1}_{D^k}} = \int_{\partial D^k} \avg{\Pi_N\LRp{\bm{\beta}u}}\cdot\bm{n}.
\]
The second term gives
\[
\LRp{\bm{\beta}\cdot\Grad_h u,\bbm{1}_{D^k}} = \LRp{\Grad u,\bm{\beta}}_{D^k} + \frac{1}{2}\LRa{\jump{u},\bm{\beta}\cdot\bm{n}} = \LRp{u,-\Grad \cdot \bm{\beta}}_{D^k} + \LRa{\avg{u},\bm{\beta}\cdot\bm{n}}
\]
through integration by parts and an assumption that $\bm{\beta}\cdot \bm{n}$ is periodic.  Cancelling volume terms, we end up with the statement of local conservation 
\[
\int_{D^k} \pd{u}{t}  + \frac{1}{2}\int_{\partial D^k} \LRp{\avg{\Pi_N\LRp{\bm{\beta}u}} + \LRp{\Pi_N\LRp{\bm{\beta}}\avg{u}}}\cdot\bm{n} = 0
\]
which is a discrete version of the continuous statement of local conservation.

Penalization can be added by adding any positive-definite stabilization term (upwind, penalty, Lax-Friedrichs) through the regular divergence flux.  

It's probably better to formulate this using continuous DG derivatives, recover flux terms, then discretize that - the flux terms \textit{should} still cancel out after discretization, right?


\section{Discrete DG derivatives}

Methods based on discrete DG derivatives also work.  

The discrete DG derivative-based method is not consistent in the sense that Galerkin orthogonality does not hold exactly.  The difference lies in the flux terms.  Assume $\Div \bm{\beta} = 0$, then
\[
\LRp{\pd{u}{t},v} + \frac{1}{2}\LRp{-{ \bm{\beta}u},\Grad_h v} + \frac{1}{2}\LRp{\bm{\beta}\cdot\Grad_h u,v} = 0.
\]
\begin{align*}
\LRp{\Grad_h\cdot \Pi_N\LRp{ \bm{\beta}u},v} &= \sum_{D^k} \LRp{-\bm{\beta}u, \Grad v}_{\LK} + \LRa{\avg{\Pi_N\LRp{\bm{\beta} u}}\cdot \bm{n},v}_{\LdK}\\
\LRp{\Grad_h u,\bm{\beta}v} &= \sum_{D^k} \LRp{-u, \Grad \cdot \LRp{\bm{\beta}v}}_{\LK} + \LRa{\bm{\beta}\cdot\bm{n}\avg{u},v}_{\LdK}.
\end{align*}
The latter term is consistent; the former is not due to the presence of $\avg{\Pi_N\LRp{\bm{\beta} u}}\cdot \bm{n}$ in the flux term.  The consistency error should then be $O(h^{N+1/2})$ using a trace inequality for $L^2$ projections.  

Can also write the inconsistent term as 
\[
\LRp{-{ \bm{\beta}u},\Grad_h v} %= -\LRp{\Grad_h v,\bm{\beta}u} = \sum_{D^k} \LRp{ v,\Grad\cdot \LRp{\bm{\beta}u}}_{D^k} - \LRa{\avg{v},\bm{\beta}\cdot \bm{n} u}
\]
but this results in the same inconsistent flux term due to the fact that $\Grad_h$ is only defined for test functions in the polynomial space.  

Note: can also use interpolants in a stable manner if using $D_h$.  Unlike SEM, this still requires an extra matvec per RHS evaluation because of the lack of diagonality of the mass matrix.  Reduces number of steps by one (no interpolation to quadrature points) but does not reduce number of total matvecs.  

For curvilinear coordinates, 



\section{Extension to other hyperbolic problems}

Example: acoustic wave equation, simply discretize by replacing $\Grad, \Grad\cdot$ with discrete versions.  Automatically skew symmetric and energy stable via integration by parts.  Also, can show why WADG works: discretize based on discrete divergence, then test with $T_{c^2}^{-1}p$ and use identities.  Note - I think this requires the use of the strictly discrete version.  

Example: Burgers' equation

Example: Kinetic energy preserving splitting of Euler (assumes exact time discretization).  Doesn't seem to help much without extra viscosity?  

Example: Entropy splitting of Buckley-Leverett?

Example: Entropy splitting of Euler (note - cannot extend to Navier-Stokes in an entropy-stable fashion due to fact that heat flux matrix is not symmetrizable w.r.t.\ homogeneous flux function, though viscous terms are.  This impacts only boundary conditions.) 

\section{Standard entropy stability estimates}

Given a nonlinear conservation law
\begin{align*}
\pd{u}{t} + \pd{f(u)}{x} = 0
\end{align*}
DG formulation is usually
\[
\LRp{\pd{u}{t},v} + \LRp{D_h f,v} = 0.
\]
Taking $v = u$ gives
\[
\frac{1}{2}\nor{u}^2 + \int_{D^k}{u D_h f} = 0.
\]
Continuous entropy stability relies on the introduction of $F,G$ such that
\[
\pd{F}{u} = u\pd{f}{u}, \qquad \pd{G}{u} = f, \qquad F = uf - G.
\]
Then, relying on product and chain rules
\[
\int_{D^k}{u \pd{f}{x}} = \int_{D^k}{\pd{(uf)}{x} - f\pd{u}{x}} = \int_{D^k}{\pd{(uf)}{x} - \pd{G}{u}\pd{u}{x}} =  \int_{D^k}{\pd{\LRp{uf-G}}{x}} = \int_{D^k}{\pd{F}{x}} = \int_{\partial D^k} F.
\]
These boundary terms can be made to cancel with appropriately defined numerical fluxes, resulting in entropy conservative schemes.  The challenge in reproducing this in the discrete case is the lack of a product and chain rule for inexact quadratures.  Jameson deals with this by introducing an integrated flux
\[
G = \frac{1}{2}\int_{-1}^1 f \LRp{ \avg{u} + \theta \jump{u}} \diff \theta.
\]
for which finite volume schemes satisfy (in a rough sense)
\[
\jump{G} = \avg{G} \jump{u}.
\]
Fischer and Carpenter (also Gassner and co-workers) use a similar idea but combine a symmetric two-point flux approximation with properties of SBP matrices to get 
\[
2\LRp{W,DF} = \LRp{W,DF} - \LRp{DW,F} + \LRa{W,F} =  \sum_i \sum_j (W_i-W_j) (MD)_{ij} F(U_i,U_j) + \LRa{W,F}.
\]

\section{Entropy splitting}

It's currently unclear how to extend the symmetric two-point flux approximation to quadrature-based DG methods.  

%Sandham and Yee use the entropy splitting 
%\[
%\eta(U) =  -\beta p^*, \qquad p^* = - \LRp{\frac{p}{\rho^\gamma}}^{\frac{1}{\beta(1-\gamma)}},
%\]
%which generates the entropy variables
%\[
%W(U) = p^*\LRp{\begin{array}{ccc}
%\frac{E}{p} - C, &\frac{-\rho u}{p}, &\frac{\rho}{p}
%\end{array}}, \qquad C = \frac{2}{\gamma-1} + (1+\beta).
%\]
%Unfortunately, the inverse mapping $U(W)$ is not explicitly known, which will greatly hamper computational efforts.  For example, this implies we cannot evolve the entropy variables explicitly, which would yield an energy estimate.  
%The Jacobian
%\[
%\pd{U}{W} = \LRs{
%d
%}, 
%\qquad \pd{W}{U} = \pd{U}{W}^{-1} = \LRs{
%}
%\]
%The flux Jacobian
%\[
%\pd{F}{U} = \LRs{\begin{array}{ccc}
%0 & u \frac{(\gamma-3)}{2} & \\
%1 & -u(\gamma-3) & 
%\end{array}}
%\]

\bibliographystyle{unsrt}
\bibliography{dg}


\end{document}



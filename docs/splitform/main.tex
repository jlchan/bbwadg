\pdfoutput=1

%\documentclass[preprint,10pt]{elsarticle}
\documentclass[preprint,10pt]{article}
%\documentclass[review]{siamart0216}
%\documentclass{siamart0216}

\usepackage{fullpage}
\usepackage{amsmath,amssymb,amsfonts,amsthm}
\theoremstyle{definition}
\newtheorem{definition}{Definition}
\theoremstyle{lemma}
\newtheorem{lemma}{Lemma}
\newtheorem*{remark}{Remark}
\theoremstyle{theorem}
\newtheorem{theorem}{Theorem}
%\usepackage{thmtools}
%\declaretheorem[style=definition,qed=$\blacksquare$,numberwithin=chapter]{definition}

\usepackage[titletoc,toc,title]{appendix}

\usepackage{array} 
\usepackage{listings}
\usepackage{mathtools}
\usepackage{pdfpages}
\usepackage[textsize=footnotesize,color=green]{todonotes}
\usepackage{bm}
\usepackage{bbm}

%\usepackage{tikz}
\usepackage[normalem]{ulem}
\usepackage{hhline}

%% ====================================== alg package
\usepackage{algorithm}
\usepackage[noend]{algpseudocode}
\usepackage{algorithmicx}
\algblock{ParFor}{EndParFor}
% customising the new block
\algnewcommand\algorithmicparfor{\textbf{parfor}}
\algnewcommand\algorithmicpardo{\textbf{do}}
\algnewcommand\algorithmicendparfor{\textbf{end\ parfor}}
\algrenewtext{ParFor}[1]{\algorithmicparfor\ #1\ \algorithmicpardo}
\algrenewtext{EndParFor}{\algorithmicendparfor}
%% ====================================== end alg package

\usepackage{graphicx}
\usepackage{subfig}
\usepackage{color}

%% ====================================== graphics

\usepackage{pgfplots}
\usepackage{pgfplotstable}
\definecolor{markercolor}{RGB}{124.9, 255, 160.65}
\pgfplotsset{width=10cm,compat=1.3}
\pgfplotsset{
tick label style={font=\small},
label style={font=\small},
legend style={font=\small}
}

\usetikzlibrary{calc}

%%% START MACRO FOR ANNOTATION OF TRIANGLE WITH SLOPE %%%.
\newcommand{\logLogSlopeTriangle}[5]
{
    % #1. Relative offset in x direction.
    % #2. Width in x direction, so xA-xB.
    % #3. Relative offset in y direction.
    % #4. Slope d(y)/d(log10(x)).
    % #5. Plot options.

    \pgfplotsextra
    {
        \pgfkeysgetvalue{/pgfplots/xmin}{\xmin}
        \pgfkeysgetvalue{/pgfplots/xmax}{\xmax}
        \pgfkeysgetvalue{/pgfplots/ymin}{\ymin}
        \pgfkeysgetvalue{/pgfplots/ymax}{\ymax}

        % Calculate auxilliary quantities, in relative sense.
        \pgfmathsetmacro{\xArel}{#1}
        \pgfmathsetmacro{\yArel}{#3}
        \pgfmathsetmacro{\xBrel}{#1-#2}
        \pgfmathsetmacro{\yBrel}{\yArel}
        \pgfmathsetmacro{\xCrel}{\xArel}

        \pgfmathsetmacro{\lnxB}{\xmin*(1-(#1-#2))+\xmax*(#1-#2)} % in [xmin,xmax].
        \pgfmathsetmacro{\lnxA}{\xmin*(1-#1)+\xmax*#1} % in [xmin,xmax].
        \pgfmathsetmacro{\lnyA}{\ymin*(1-#3)+\ymax*#3} % in [ymin,ymax].
        \pgfmathsetmacro{\lnyC}{\lnyA+#4*(\lnxA-\lnxB)}
        \pgfmathsetmacro{\yCrel}{\lnyC-\ymin)/(\ymax-\ymin)} % THE IMPROVED EXPRESSION WITHOUT 'DIMENSION TOO LARGE' ERROR.

        % Define coordinates for \draw. MIND THE 'rel axis cs' as opposed to the 'axis cs'.
        \coordinate (A) at (rel axis cs:\xArel,\yArel);
        \coordinate (B) at (rel axis cs:\xBrel,\yBrel);
        \coordinate (C) at (rel axis cs:\xCrel,\yCrel);

        % Draw slope triangle.
        \draw[#5]   (A)-- node[pos=0.5,anchor=north] {1}
                    (B)-- 
                    (C)-- node[pos=0.5,anchor=west] {#4}
                    cycle;
    }
}
%%% END MACRO FOR ANNOTATION OF TRIANGLE WITH SLOPE %%%.

\newcommand{\logLogSlopeTriangleNeg}[5]
{
    % #1. Relative offset in x direction.
    % #2. Width in x direction, so xA-xB.
    % #3. Relative offset in y direction.
    % #4. Slope d(y)/d(log10(x)).
    % #5. Plot options.

    \pgfplotsextra
    {
        \pgfkeysgetvalue{/pgfplots/xmin}{\xmin}
        \pgfkeysgetvalue{/pgfplots/xmax}{\xmax}
        \pgfkeysgetvalue{/pgfplots/ymin}{\ymin}
        \pgfkeysgetvalue{/pgfplots/ymax}{\ymax}

        % Calculate auxilliary quantities, in relative sense.
        \pgfmathsetmacro{\xArel}{#1}
        \pgfmathsetmacro{\yArel}{#3}
        \pgfmathsetmacro{\xBrel}{#1-#2}
        \pgfmathsetmacro{\yBrel}{\yArel}
        \pgfmathsetmacro{\xCrel}{\xArel}

        \pgfmathsetmacro{\lnxB}{\xmin*(1-(#1-#2))+\xmax*(#1-#2)} % in [xmin,xmax].
        \pgfmathsetmacro{\lnxA}{\xmin*(1-#1)+\xmax*#1} % in [xmin,xmax].
        \pgfmathsetmacro{\lnyA}{\ymin*(1-#3)+\ymax*#3} % in [ymin,ymax].
        \pgfmathsetmacro{\lnyC}{\lnyA+#4*(\lnxA-\lnxB)}
        \pgfmathsetmacro{\yCrel}{\lnyC-\ymin)/(\ymax-\ymin)} % THE IMPROVED EXPRESSION WITHOUT 'DIMENSION TOO LARGE' ERROR.

        % Define coordinates for \draw. MIND THE 'rel axis cs' as opposed to the 'axis cs'.
        \coordinate (A) at (rel axis cs:\xArel,\yArel);
        \coordinate (B) at (rel axis cs:\xBrel,\yBrel);
        \coordinate (C) at (rel axis cs:\xCrel,\yCrel);

        % Draw slope triangle.
        \draw[#5]   (A)-- node[pos=.5,anchor=south] {1}
                    (B)-- 
                    (C)-- node[pos=0.5,anchor=west] {#4}
                    cycle;
    }
}
%%% END MACRO FOR ANNOTATION OF TRIANGLE WITH SLOPE %%%.

%%% START MACRO FOR ANNOTATION OF TRIANGLE WITH SLOPE %%%.
\newcommand{\logLogSlopeTriangleFlipNeg}[5]
{
    % #1. Relative offset in x direction.
    % #2. Width in x direction, so xA-xB.
    % #3. Relative offset in y direction.
    % #4. Slope d(y)/d(log10(x)).
    % #5. Plot options.

    \pgfplotsextra
    {
        \pgfkeysgetvalue{/pgfplots/xmin}{\xmin}
        \pgfkeysgetvalue{/pgfplots/xmax}{\xmax}
        \pgfkeysgetvalue{/pgfplots/ymin}{\ymin}
        \pgfkeysgetvalue{/pgfplots/ymax}{\ymax}

        % Calculate auxilliary quantities, in relative sense.
        %\pgfmathsetmacro{\xArel}{#1}
        %\pgfmathsetmacro{\yArel}{#3}
        \pgfmathsetmacro{\xBrel}{#1-#2}
        \pgfmathsetmacro{\yBrel}{#3}
        \pgfmathsetmacro{\xCrel}{#1}

        \pgfmathsetmacro{\lnxB}{\xmin*(1-(#1-#2))+\xmax*(#1-#2)} % in [xmin,xmax].
        \pgfmathsetmacro{\lnxA}{\xmin*(1-#1)+\xmax*#1} % in [xmin,xmax].
        \pgfmathsetmacro{\lnyA}{\ymin*(1-#3)+\ymax*#3} % in [ymin,ymax].
        \pgfmathsetmacro{\lnyC}{\lnyA+#4*(\lnxA-\lnxB)}
        \pgfmathsetmacro{\yCrel}{\lnyC-\ymin)/(\ymax-\ymin)} % THE IMPROVED EXPRESSION WITHOUT 'DIMENSION TOO LARGE' ERROR.

	\pgfmathsetmacro{\xArel}{\xBrel}
        \pgfmathsetmacro{\yArel}{\yCrel}

        % Define coordinates for \draw. MIND THE 'rel axis cs' as opposed to the 'axis cs'.
        \coordinate (A) at (rel axis cs:\xArel,\yArel);
        \coordinate (B) at (rel axis cs:\xBrel,\yBrel);
        \coordinate (C) at (rel axis cs:\xCrel,\yCrel);

        % Draw slope triangle.
        \draw[#5]   (A)-- node[pos=0.5,anchor=east] {#4}
                    (B)-- 
                    (C)-- node[pos=0.5,anchor=north] {1}
                    cycle;
    }
}
%%% END MACRO FOR ANNOTATION OF TRIANGLE WITH SLOPE %%%.


%%% START MACRO FOR ANNOTATION OF TRIANGLE WITH SLOPE %%%.
\newcommand{\logLogSlopeTriangleFlip}[5]
{
    % #1. Relative offset in x direction.
    % #2. Width in x direction, so xA-xB.
    % #3. Relative offset in y direction.
    % #4. Slope d(y)/d(log10(x)).
    % #5. Plot options.

    \pgfplotsextra
    {
        \pgfkeysgetvalue{/pgfplots/xmin}{\xmin}
        \pgfkeysgetvalue{/pgfplots/xmax}{\xmax}
        \pgfkeysgetvalue{/pgfplots/ymin}{\ymin}
        \pgfkeysgetvalue{/pgfplots/ymax}{\ymax}

        % Calculate auxilliary quantities, in relative sense.
        %\pgfmathsetmacro{\xArel}{#1}
        %\pgfmathsetmacro{\yArel}{#3}
        \pgfmathsetmacro{\xBrel}{#1-#2}
        \pgfmathsetmacro{\yBrel}{#3}
        \pgfmathsetmacro{\xCrel}{#1}

        \pgfmathsetmacro{\lnxB}{\xmin*(1-(#1-#2))+\xmax*(#1-#2)} % in [xmin,xmax].
        \pgfmathsetmacro{\lnxA}{\xmin*(1-#1)+\xmax*#1} % in [xmin,xmax].
        \pgfmathsetmacro{\lnyA}{\ymin*(1-#3)+\ymax*#3} % in [ymin,ymax].
        \pgfmathsetmacro{\lnyC}{\lnyA+#4*(\lnxA-\lnxB)}
        \pgfmathsetmacro{\yCrel}{\lnyC-\ymin)/(\ymax-\ymin)} % THE IMPROVED EXPRESSION WITHOUT 'DIMENSION TOO LARGE' ERROR.

	\pgfmathsetmacro{\xArel}{\xBrel}
        \pgfmathsetmacro{\yArel}{\yCrel}

        % Define coordinates for \draw. MIND THE 'rel axis cs' as opposed to the 'axis cs'.
        \coordinate (A) at (rel axis cs:\xArel,\yArel);
        \coordinate (B) at (rel axis cs:\xBrel,\yBrel);
        \coordinate (C) at (rel axis cs:\xCrel,\yCrel);

        % Draw slope triangle.
        \draw[#5]   (A)-- node[pos=0.5,anchor=east] {#4}
                    (B)-- 
                    (C)-- node[pos=0.5,anchor=south] {1}
                    cycle;
    }
}
%%% END MACRO FOR ANNOTATION OF TRIANGLE WITH SLOPE %%%.



\usepackage{stmaryrd}


\renewcommand{\topfraction}{0.85}
\renewcommand{\textfraction}{0.1}
\renewcommand{\floatpagefraction}{0.75}

\newcommand{\vect}[1]{\ensuremath\boldsymbol{#1}}
\newcommand{\tensor}[1]{\underline{\bm{#1}}}
\newcommand{\del}{\triangle}
\newcommand{\curl}{\grad \times}
\renewcommand{\div}{\grad \cdot}

\newcommand{\bbm}[1]{\mathbbm{#1}}
\newcommand{\bs}[1]{\boldsymbol{#1}}
\newcommand{\equaldef}{\stackrel{\mathrm{def}}{=}}

\newcommand{\td}[2]{\frac{{\rm d}#1}{{\rm d}{\rm #2}}}
\newcommand{\pd}[2]{\frac{\partial#1}{\partial#2}}
\newcommand{\pdd}[2]{\frac{\partial^2#1}{\partial#2^2}}
\newcommand{\pdn}[3]{\frac{\partial^{#3}#1}{\partial#2^{#3}}}
\newcommand{\mb}[1]{\mathbf{#1}}
\newcommand{\mbb}[1]{\mathbb{#1}}
\newcommand{\mc}[1]{\mathcal{#1}}
\newcommand{\snor}[1]{\left| #1 \right|}
\newcommand{\nor}[1]{\left\| #1 \right\|}
\newcommand{\LRp}[1]{\left( #1 \right)}
\newcommand{\LRs}[1]{\left[ #1 \right]}
\newcommand{\LRa}[1]{\left\langle #1 \right\rangle}
\newcommand{\LRb}[1]{\left| #1 \right|}
\newcommand{\LRc}[1]{\left\{ #1 \right\}}
\newcommand{\LRceil}[1]{\left\lceil #1 \right\rceil}
\newcommand{\LRl}[1]{\left. #1 \right|}

%\newcommand{\cond}[1]{\kappa\LRp{#1}}
\newcommand{\cond}[2]{\nor{#1}_{#2}\nor{{#1}^{-1}}_{#2}}


\newcommand{\Grad} {\ensuremath{\nabla}}
\newcommand{\Div} {\ensuremath{\nabla\cdot}}
\newcommand{\jump}[1] {\ensuremath{\llbracket#1\rrbracket}}
\newcommand{\avg}[1] {\ensuremath{\LRc{\!\{#1\}\!}}}

\newcommand{\Oh}{{\Omega_h}}
\renewcommand{\L}{L^2\LRp{\Omega}}
\newcommand{\LK}{L^2\LRp{D^k}}
\newcommand{\LdK}{L^2\LRp{\partial D^k}}
\newcommand{\Dhat}{\widehat{D}}
\newcommand{\Lhat}{L^2\LRp{\Dhat}}

\newcommand{\eval}[2][\right]{\relax
  \ifx#1\right\relax \left.\fi#2#1\rvert}

\def\etal{{\it et al.~}}


\newcommand{\note}[1]{{\color{blue}{#1}}}


\newcommand{\LinfDk}{L^{\infty}\LRp{D^k}}

\newcommand{\diag}[1]{{\rm diag}\LRp{#1}}

\newcommand{\Ksub}{K_{\rm sub}}

\newcolumntype{C}[1]{>{\centering\let\newline\\\arraybackslash\hspace{0pt}}m{#1}}

%% d in integrand
\newcommand*\diff[1]{\mathop{}\!{\mathrm{d}#1}}


\makeatletter
\renewcommand\d[1]{\mspace{6mu}\mathrm{d}#1\@ifnextchar\d{\mspace{-3mu}}{}}
\makeatother

\date{}
\author{Jesse Chan}
\title{On discrete entropy conservation for general high order discontinuous Galerkin methods}

\begin{document}

\maketitle

\begin{abstract}
High order methods based on diagonal-norm summation by parts operators can be shown to semi-discretely conserve entropy for nonlinear systems of hyperbolic PDEs \cite{fisher2013high,carpenter2014entropy}.  These include discontinuous Galerkin spectral element methods (DG-SEM) \cite{gassner2016split,gassner2016well,wintermeyer2017entropy, chen2017entropy} based on mass-lumping.  In this work, we describe how to extend discretely entropy conservative schemes for a more general class of DG methods using quadrature-based projections and global differentiation operators.  The construct of these methods is independent of the choice of basis and accuracy of quadrature rule used, and includes DG methods involving generalized SBP operators without boundary nodes and certain dense-norm SBP operators.  Numerical experiments confirm the stability and high order accuracy of the proposed methods for the Burgers, shallow water, and compressible Euler equations.  
\end{abstract}

\tableofcontents

\section{Introduction}

\note{High order methods are good boilerplate}

\note{High order methods are not stable for nonlinear problems.  Stability of discretizations is a hot topic: artificial viscosity, polynomial de-aliasing,  \cite{karniadakis2013spectral}}

\cite{tadmor1987numerical, tadmor2003entropy}

Hughes entropy variables \cite{hughes1986new}

\cite{hicken2016multidimensional}

\cite{gassner2016split,gassner2016well,wintermeyer2017entropy}

\cite{chen2017entropy}

While DG-SEM based on flux differencing is entropy conservative, this is tied to a specific choice of basis and quadrature.  Furthermore, GLL quadratures and diagonal-norm SBP operators do not naturally generalize to triangles or tetrahedra \cite{helenbrook2009existence}.  Diagonal-norm SBP operators can be constructed for triangles and tetrahedra \cite{chin1999higher, cohen2001higher, hicken2016multidimensional, chen2017entropy}; however, the number of nodal points for such operators is greater than the dimension of the natural polynomial approximation space.  Furthermore, appropriate point sets have only been constructed for $N \leq 4$ in three dimensions \cite{zhebel2014comparison}.  

\section{Entropy conservative collocation DG methods}

\subsection{Entropy stability for systems of hyperbolic PDEs}

We consider systems of nonlinear conservation laws in $\mathbb{R}^d$ with $n$ variables
\begin{align}
\pd{\bm{u}}{t} + \sum_{k=1}^d \pd{\bm{f_k(\bm{u})}}{\bm{x}_k} &= 0, \qquad \bm{u}(\bm{x},t) = (u_1(\bm{x},t),\ldots,u_n(\bm{x},t)).
\label{eq:pde}
\end{align}
where the fluxes $f_k(\bm{u})$ are smooth functions of the vector of conservative variables $\bm{u}(\bm{x},t)$.
The Jacobian matrix $\bm{A}_k(\bm{u})$ is defined entrywise as
\[
\LRp{\bm{A}_k(\bm{u})}_{ij} = \LRp{\pd{\bm{f}_k(\bm{u})}{\bm{u}_j}}_i.
\]
We are interested in systems for which there exists a convex entropy function $U(\bm{u})$ such that  
\begin{equation}
U''(\bm{u})\bm{A}_k(\bm{u}) = \LRp{U''(\bm{u}) \bm{A}_k(\bm{u})}^T.
\label{eq:entropysym}
\end{equation}
For systems with convex entropy functions, one can define entropy variables $\bm{v} = U'(\bm{u})$.  The convexity of the $U(\bm{u})$ guarantees that the mapping between conservative and entropy variables is invertible.  

It can be shown (see, for example, \cite{mock1980systems}) that (\ref{eq:entropysym}) is equivalent to the existence of an entropy flux functions $F_k(\bm{u})$ such that
\[
\bm{v}^T \pd{\bm{f}_k}{\bm{u}} = \pd{F_k(\bm{u})}{\bm{u}}^T.
\]
The flux function and entropy flux function are further related through the so-called entropy potential $\psi_k$ 
\[
\psi_k(\bm{v}) = \bm{v}^T\bm{f}_k(\bm{u}(\bm{v})) - F_k(\bm{u}(\bm{v})), \qquad \psi_k'(\bm{v}) = \bm{f}_k(\bm{u}(\bm{v})).
\]

When $\bm{u}$ is smooth, multiplying (\ref{eq:pde}) on the left by $\bm{v}^T = U'(\bm{u})^T$, applying the definition of the entropy flux and using the chain rule yields the conservation of entropy
\[
\pd{U(\bm{u})}{t} + \sum_{k=1}^d \pd{F_k(\bm{u})}{\bm{x}_k} = 0.
\]
More generally, it can be shown that physically relevant solutions of (\ref{eq:pde}) (defined as the limiting solution for an appropriately defined vanishing viscosity) satisfy the entropy inequality
\[
\pd{U(\bm{u})}{t} + \sum_{k=1}^d \pd{F_k(\bm{u})}{\bm{x}_k} \leq 0.  
\]
%Integration over $\mathbb{R}^d$ then shows that the entropy of the solution decreases in time.  
When $d = 1$, integrating the entropy inequality over $[-1,1]$ and using the definition of the entropy potential yields
\begin{equation}
\int_{-1}^1 \pd{U(\bm{u})}{t} + \left.\LRp{\bm{v}^T\bm{f}_k(\bm{u}(\bm{v}))-\psi(\bm{v})}\right|_{-1}^1 \leq 0.  
\label{eq:consentropy}
\end{equation}

%\note{Wave equation example}


\subsection{Entropy conservative collocation methods based on SBP operators}

Let $\widehat{x}_i \in [-1,1]$ denote the $(N+1)$ point Gauss-Lobatto-Legendre (GLL) quadrature rule with positive weights $w_1,\ldots,w_{N+1}$.  Let $\ell_i(x)$ denote the nodal Lagrange basis defined at GLL points, and define the differentiation matrix $\bm{D}$ be defined as 
\[
\bm{D}_{ij} = \pd{\ell_j(x_i)}{x}.
\]
The matrix $\bm{D}$ maps from nodal values of a polynomial to nodal values of its derivative.  We also define the mass matrix $\bm{M}$
\[
\bm{M}_{ij} = \int_{-1}^1 \ell_i(x)\ell_j(x)\diff{x} \approx \delta_{ij} w_i,
\]
where the approximation results evaluating the integral using GLL quadrature.  For this choice of quadrature, the Kronecker property $\ell_j(x_i) = \delta_{ij}$ gives that $\bm{M}$ is a diagonal matrix.  

The mass and derivative matrices may be used to define the diagonal norm SBP operator $\bm{S} = \bm{M}\bm{D}$, with the summation-by-parts property 
\[
\bm{S} = \bm{B} - \bm{S}^T, \qquad \bm{B}_{ij} = \begin{cases}
1 &i = j = 1 \\
1 &i = j = N+1 \\
0 &\text{otherwise}
\end{cases}.
\]
This property is simply a restatement of integration by parts, using that the derivative of $\ell_j(x) \in P^{N-1}$ and that GLL quadrature is exact for polynomials of degree $2N-1$.  

%In order to simplify notation for flux differencing, we define the following bilinear matrix product 
%\[
%\LRp{\bm{A} \odot \bm{f}_S(\bm{x},\bm{y})}_i = \sum_j \bm{A}_{ij} \bm{f}_S(\bm{x}_j,\bm{y}_i).  
%\]
We seek a nodal approximation $\bm{u}_i \approx \bm{u}(x_i)$ to the solution $\bm{u}(x)$ of (\ref{eq:pde}).  A typical nodal collocation method in conservation form is given as follows 
\[
\pd{\bm{u}_i}{t} + \sum_{j=1}^{N+1}\LRp{\LRp{\bm{D}\otimes \bm{I}}_{ij}\bm{f}(\bm{u}_j) +\LRp{\bm{M}^{-1}\bm{B}}_{ij}(\bm{f}_i^* - \bm{f}(\bm{u}_i))} = 0.
\]
Here, $\bm{f}^*$ is a numerical flux used for the weak imposition of boundary or continuity conditions \cite{hesthaven2007nodal}.  This method can be derived from an appropriate DG formulation under a specific choice of quadrature.  

This formulation conserves the quantities $\bm{u}_i$, but (except in specific cases) does not satisfy a discrete analogue of the statement of entropy conservation (\ref{eq:consentropy}).  Entropy conservative schemes may be constructed based on the SBP property and a ``flux differencing'' approach \cite{gassner2016split} involving a definition of entropy conservative/stable fluxes given by Tadmor \cite{tadmor1987numerical, tadmor2003entropy}.  
\begin{definition}
Let $\bm{f}_S(\bm{u}_L,\bm{u}_R)$ be a bivariate function such that it is symmetric and consistent with the flux function $\bm{f}(\bm{u})$
\[
\bm{f}_S(\bm{u}_L,\bm{u}_R) = \bm{f}_S(\bm{u}_R,\bm{u}_L), \qquad \bm{f}_S(\bm{u},\bm{u}) = \bm{f}(\bm{u})
\]
The numerical flux $\bm{f}_S(\bm{u}_L, \bm{u}_R)$ is entropy conservative if, for entropy variables $\bm{v}_L = \bm{v}(\bm{u}_L), \bm{v}_R = \bm{v}(\bm{u}_R)$
\[
\LRp{\bm{v}_L - \bm{v}_R}^T \bm{f}_S(\bm{u}_L,\bm{u}_R) = (\psi_L - \psi_R), \qquad \psi_L = \psi(\bm{v}(\bm{u}_L)), \quad \psi_R = \psi(\bm{v}(\bm{u}_R)).  
\]
Similarly, a flux $\bm{f}_S(\bm{u}_L, \bm{u}_R)$ is referred to as entropy stable if $\LRp{\bm{v}_L - \bm{v}_R}^T \bm{f}_S(\bm{u}_L,\bm{u}_R) \leq (\psi_L - \psi_R)$.
\label{def:tadmor}
\end{definition}
An entropy conservative nodal collocation method can then be derived by modifying the flux derivative as follows
\begin{equation}
\pd{\bm{u}_i}{t} + \sum_{j=1}^{N+1}\LRp{\LRp{2\bm{D}\otimes \bm{I}}_{ij}\bm{f}_S(\bm{u}_i,\bm{u}_j) +\LRp{\bm{M}^{-1}\bm{B}}_{ij}(\bm{f}_i^* - \bm{f}(\bm{u}_i))}= 0.
\label{eq:dgsem}
\end{equation}
The replacement of the derivative of the flux function with the flux differencing allows one to prove that (\ref{eq:dgsem}) conserves a discrete entropy \cite{gassner2017br1,chen2017entropy}.  We reproduce these proofs here, as we will refer to them in following sections.  
\begin{theorem}
The collocation method (\ref{eq:dgsem}) is discretely entropy conservative in the sense that
\[
\bm{1}^T\LRp{\bm{M}\pd{U(\bm{u})}{t} + \bm{B}\LRp{\bm{v}^T\bm{f}^* - \bm{\psi}}} = 0,
\]
where $\bm{1}$ denotes the vector of ones and $\bm{\psi}_i = \psi(\bm{v}_i)$ is the entropy potential in terms of the nodal value of $\bm{u}_i$.  
\label{thm:dgsem}
\end{theorem}
\begin{proof}
Let $\bm{v}_i = \bm{v}(\bm{u}_i)$ denote the entropy variables, defined in terms of the nodal values $\bm{u}_i$.  We multiply (\ref{eq:dgsem}) by $\LRp{\bm{M}\bm{v}}^T$.  
Using the fact that $\bm{M}$ is diagonal, the time derivative term can be manipulated to 
\[
\bm{v}^T\bm{M}\pd{\bm{u}}{t} = \sum_{i=1}^{N+1} \bm{M}_{ii} \bm{v}_i^T \pd{\bm{u}_i}{t} = \sum_{i=1}^{N+1} \bm{M}_{ii} \pd{S_i}{t} = \bm{1}^T\bm{M}\pd{S_i}{t},
\]
where we have used continuity in time to arrive at the pointwise relation 
\[
\bm{v}^T \pd{\bm{u}}{t} = \pd{U(\bm{u})}{\bm{u}}^T \pd{\bm{u}}{t} = \pd{U(\bm{u})}{t}.
\]

The spatial derivative term can be manipulated using the summation by parts property $\bm{M}\bm{D} = \bm{S} = \bm{B}-\bm{S}^T$
\begin{align}
\sum_{i,j=1}^{N+1} \bm{v}_i^T 2 \bm{M}\bm{D}_{ij}\bm{f}_S(\bm{u}_i,\bm{u}_j) &= \sum_{i,j=1}^{N+1} \bm{v}_i^T (\bm{S} + \bm{B} - \bm{S}^T)_{ij}\bm{f}_S(\bm{u}_i,\bm{u}_j) \nonumber\\
&= \sum_{i,j=1}^{N+1} \bm{S}_{ij}\bm{v}_i^T \bm{f}_S(\bm{u}_i,\bm{u}_j) - \bm{S}_{ji}\bm{v}_i^T \bm{f}_S(\bm{u}_i,\bm{u}_j) + \bm{B}_{ij}\bm{v}_i^T \bm{f}_S(\bm{u}_i,\bm{u}_j). \label{eq:sbpentropy}
\end{align}
Rearranging indices and using symmetry of $\bm{f}_S(\bm{u}_i,\bm{u}_j)$ then yields that
\begin{align*}
\sum_{i,j=1}^{N+1} \bm{S}_{ij}\bm{v}_i^T \bm{f}_S(\bm{u}_i,\bm{u}_j) - \bm{S}_{ji}\bm{v}_i^T \bm{f}_S(\bm{u}_i,\bm{u}_j) &= \sum_{i,j=1}^{N+1} \bm{S}_{ij} \LRp{\bm{v}_i - \bm{v}_j}^T\bm{f}_S(\bm{u}_i,\bm{u}_j)\\
&= \sum_{i,j=1}^{N+1} \bm{S}_{ij} \LRp{\bm{\psi}_i-\bm{\psi}_j} = -\bm{1}^T \bm{S}\bm{\psi} = -\bm{1}^T\bm{B}\bm{\psi},
\end{align*}
where we have used that $\bm{S}\bm{1} = 0$ and the summation by parts property in the last line.  

Using that $\bm{B}$ is diagonal, combining the latter term in (\ref{eq:sbpentropy}) with boundary terms involving the numerical flux yields
\[
\sum_{i,j = 1}^{N+1} \bm{B}_{ij}\bm{v}_i^T \bm{f}_S(\bm{u}_i,\bm{u}_j) + \bm{v}_i^T\bm{B}_{ij}\LRp{\bm{f}_i^* - \bm{f}(\bm{u}_i)} = \bm{1}^T\bm{B} \LRp{\bm{v}^T \bm{f}^*},
\]
where we have used the consistency of $\bm{f}_S(\bm{u},\bm{u}) = \bm{f}(\bm{u})$.  Combining everything results in a discrete statement of the conservation of entropy (\ref{eq:consentropy}).   
%Combining everything results in a discrete statement of the conservation of entropy (\ref{eq:consentropy}) 
%\[
%\bm{1}^T\LRp{\bm{M}\pd{U(\bm{u})}{t} + \bm{B}\LRp{\bm{v}^T\bm{f}^* - \bm{\psi}}} = 0.
%\]
\end{proof}

\begin{remark}
The proof of entropy conservation can be extended beyond a single domain by specifying appropriate simultaneous approximation terms (SBP-SAT).  This can be done by taking $\bm{f}_i^*$ to be the entropy conservative flux $f_S(u_L,u_R)$ at the interface between two elements \cite{carpenter2014entropy,gassner2017br1,chen2017entropy}.  Similarly, employing an entropy stable flux at element interfaces results in an entropy stable method which satisfies a global entropy inequality.  
\end{remark}


\section{Entropy conservative general DG methods}

In this section, we focus on constructing entropy conservative DG schemes with arbitrary bases and quadrature rules.  These include dense norm SBP operators and generalized SBP (GSBP) operators which do not contain boundary points, which introduce additional difficulties in constructing entropy conservative DG schemes.  These difficulties present themselves primarily in three areas:
\begin{enumerate}
\item It is unclear how to generalize the flux differencing approach to arbitrary choices of quadrature.  
\item The proof of entropy stability relies on a diagonal norm mass matrix, which commutes with pointwise multiplication by the (nodal values of) the entropy variables.  
\item Entropy stable formulations based on dense norm or GSBP operators require non-trivial SBP-SAT terms \cite{ranocha2017extended}.  These can be derived by hand for simple problems such as Burgers' equation, but can be difficult to determine for more complex systems of nonlinear PDEs.  
\end{enumerate}

We simultaneously address all three areas of difficulty by introducing a generalization of the entropy conservative DG-SEM scheme (\ref{eq:dgsem}) to arbitrary quadratures and choices of basis.  Additionally, we prove that the generalized scheme is entropy conservative (stable) on multiple elements in the following sections.  

%We will get around these difficulties for dense-norm DG methods by satisfying three requirements for entropy stability
%\begin{itemize}
%\item $\bm{D}\bm{1} = 0$ for use of the flux formulation.  
%\item For some norm matrix $\bm{W}$, $\bm{W}\bm{D} = \bm{B}-\bm{D}^T\bm{W}$ where $\bm{B}$ is a suitably defined boundary operator.
%\item Contraction with the \textit{projection} of the entropy variables.  
%\end{itemize}

\subsection{A continuous interpretation of flux differencing}

We define the quadrature-based $L^2$ projection $\Pi_N$ as the $L^2$ projection where integrals are evaluated using some (inexact) quadrature.  We also define the discrete approximation space $V_h$, and assume that the quadrature is sufficiently accurate such that integration by parts holds for any two elements of $V_h$.  Finally, we assume that the discrete derivative (to be defined) maps from $V_h$ to $V_h$.  

The split-form methodology of Gassner et al.\ can be interpreted as the following: let $f_S(u_L,u_R)$ be a symmetric and consistent two-point flux (i.e. $f_S(u,u) = f(u)$).  Split forms can be rewritten at the continuous level as replacing the flux derivative with differentiation along one argument of $f_S(u_L,u_R)$, then evaluation of the other argument at $u$
\[
\pd{f(u)}{x} \Longrightarrow 2\left.\pd{f_S(u(x),u(y))}{x}\right|_{y=x}.
\]
This allows for the recovery of different split-forms.  For example, for Burgers equation, choosing $f_S(u_L,u_R) = (u_L^2 + u_Lu_R + u_R^2)/6$ gives
\[
2\left.\pd{f_S(u(x),u(y))}{x}\right|_{y=x} = \frac{1}{3}\left.\pd{\LRp{u(x)^2 + u(x)u(y) + u(y)^2}}{x}\right|_{y=x} = \frac{1}{3}\LRp{\pd{u^2}{x} + u\pd{u}{x}}
\]
due to the fact that $\pd{u(y)^2}{x} = 0$.  The resulting form of Burgers equation appears in the intermediate steps of energy stability proofs, and is the basis for discretely stable DG methods.  When applied discretely (using quadrature-based projections), the resulting right hand side for Burgers' equation becomes 
\[
\LRp{\frac{1}{3}\LRp{\pd{\LRp{\Pi_N u^2}}{x} + \Pi_N \LRp{u\pd{u}{x}}}, v}.
\]
Using the fact integration by parts holds when considering two elements of $V_h$, and that $\LRp{\Pi_N u^2, v} = \LRp{u^2,v}$ for any $v \in V_h$, it can be shown that the volume terms cancel out for $v = u$, and we're left with boundary terms (which is a prerequisite of energy/entropy stability).  

We note that the recovery of the split-form of Burgers' equation in the above example relies only on the property that 
\[
\pd{f(u(y))}{x} = f(u(y)) \pd{1}{x} = 0.
\]
As such, replacing $\pd{}{x}$ by any differential operator $D$ such that $D 1 = 0$ will recover a similar formulation.

This approach yields the following evaluation (in Matlab for lack of better notation): let 
\[
\verb|fS = @(uL,uR) uL.^2+uL.*uR+uR.^2|
\]
be the flux function.  Let \verb+uq = Vq * u+ denote the interpolation of $u$ to quadrature points, and let \verb+Dx+ denote the differentiation matrix.  Define \verb+Pq = M\(Vq' * diag(wq))+, where \verb+wq+ are quadrature weights, and let \verb+[ux uy] = meshgrid(uq)+.  Then, assuming element-constant Jacobians, the split form can be evaluated via
\[
\verb+Pq * diag((Vq*Dx*Pq) * fS(ux, uy)) = Pq * sum((Vq* Dx* Pq) .* fS(ux, uy), 2).+
\]
The latter expression can be performed by computing \verb+fS(ux,uy)+ on the fly, avoiding the need to store it as a matrix.  It can also be generalized to handle non-square matrices in a very similar way.

\subsection{Discretely entropy conservative general DG methods on a single domain}

Global DG operators reveal how to construct SBP-SAT terms for any collocation-type DG methods (where $L^2$ projection and interpolation are equivalent).  However, for quadrature-based DG methods, it is not possible to test with the entropy variables directly because the test space is polynomial.  However, we may test with the projections of the entropy variables.  On the left hand side, we may use the fact that, under a method of lines discretization, 
\[
u(x,t) = \sum_{j=1}^{N_p} u_j(t) \phi_j(x), \qquad \pd{u}{t} \in P^N.
\]
Thus, testing with $\Pi_N v$ yields
\[
\int_{D^k} \pd{u}{t}\Pi_N v = \int_{D^k} \pd{u}{t} v   
\]
which yields a pointwise relationship which can be manipulated into the time derivative of entropy.  

The right-hand side is more difficult.  For collocation DG methods, interpolation and quadrature-based $L^2$ projection are identical.  However, when using a quadrature rule with $N_q > N_p$, this is not the case.  The solution is to evaluate the flux with $u(\Pi_N v)$ - in other words, evaluate the flux variables in terms of the projections of the entropy variables.  

\subsection{SBP-SAT terms by global DG differentiation operators}

We can define a global discontinuous Galerkin operator $\bm{D}^x_h: P^N\rightarrow P^N$ such that
\[
\sum_k \LRp{D^x_h u,v}_{D^k} = \sum_k \LRp{\pd{u}{x},v}_{D^k} + \LRa{\frac{1}{2}\jump{u}\bm{n}_x,v}_{\partial D^k}.
\]
Then, composing $D^x_h \Pi_N$ gives global skew-symmetry $\LRp{D^x_h\Pi_N u, v}_{\L} = \LRp{- D^x_h\Pi_N v, u}_{\L}$ for all $u,v\in \L$.  The downside to this simple construction is that the stencil of the DG scheme now incorporates all degrees of freedom on neighboring elements due to the composition with the projection.  

We adapt the construction of local differentiation operators by Shi and Shu to construct a global DG differentiation operator.  
\[
2\LRp{D^x_h u,v} = \sum_k \LRp{2\pd{ \Pi_N u}{x},v}_{D^k} + \LRa{\LRp{u_P - \Pi_N u_M}\bm{n}_x, v} + \LRa{\LRp{u_M - \Pi_Nu_M }\bm{n}_x,\Pi_Nv}.
\]
In skew-symmetric form, this gives 
\[
2\LRp{D^x_h u,v} = \sum_k \LRp{\pd{ \Pi_N u}{x},v}_{D^k} - \LRp{u,\pd{\Pi_N v}{x}}_{D^k} + \LRa{\LRp{u_P - \Pi_N u_M}\bm{n}_x, v} + \LRa{{u_M }\bm{n}_x,\Pi_Nv}.
\]
Taking $v = u_M$ and cancelling $u_P u_M \bm{n}_x$ terms shows that the differentiation operator has the global SBP property and is skew-symmetric  w.r.t.\ a discrete $L^2$ inner product.  The coupling requires only traces of neighboring elements.  

It's currently unclear how to distinguish whether different operators are high order accurate.  


\subsection{Conservation of secondary properties}

It holds but with respect to specifically defined fluxes.  

\section{Numerical experiments}

\subsection{Burgers}

\subsection{Shallow water}

We describe in this section the shallow water equations in one dimension.  We use this model problem to illustrate the proposed method because (unlike Burgers' equation) it highlights some non-trivial details in proving entropy stability associated with the nonlinear equations of compressible flow.  

The one-dimensional shallow water equations are 
\begin{align*}
\pd{h}{t} + \pd{hv}{x} &= 0\\
\pd{hv}{t} + \pd{\LRp{hv^2 + gh^2/2}}{x} + gh\pd{b}{x}&= 0.
\end{align*}
where $h$ is the water height, $v$ is the velocity, $g$ is the gravitational constant, and $b$ is bottom topography.  

The entropy associated with this system is the total energy $e = k + p$, where 
\[
k = hv^2/2, \qquad p = gh^2/2 + ghb.
\]
Testing with the entropy variables 
\[
q_1 = \pd{e}{h} = g(h+b)-v^2/2, \qquad q_2 = \pd{e}{hv} = v
\]
recovers a statement of conservation of energy
\[
\pd{e}{t} + \pd{}{x}\LRp{hv^3/2 + g(hv)(h+b)} = 0.  
\]

This conservation of energy does not hold at the discrete level for two reasons.  The first reason is because, under inexact quadrature, calculus identities such as the chain or product rule do not hold, and naive spatial formulations can not be manipulated into a form which is amenable to a proof of discrete entropy stability.  This issue can be remedied by the use of a split formulation (as shown in Gassner et al.) which is provably entropy stable under inexact quadrature.  

The second more subtle reason why entropy conservation does not hold at the discrete level is because the entropy variables are not contained within the test space.  Moreover, the spatial part of the proposed discretization requires one discrete definition of the entropy variables for entropy stability, while the temporal part of the discretization requires another discrete definition of the entropy variables.  We address this by adopting a rescaling in the mass matrix on the left hand side involving both definitions of the discrete entropy variables.  Under an appropriate choice of basis and quadrature, the proposed approach recovers existing DG-SEM discretizations. 


\subsection{Compressible Euler equations}

\section{Conclusions}

\section{Acknowledgments}



\bibliographystyle{unsrt}
\bibliography{dg}


\end{document}



\pdfoutput=1

%\documentclass[preprint,10pt]{elsarticle}
\documentclass[preprint,10pt]{article}
%\documentclass[review]{siamart0216}
%\documentclass{siamart0216}

\usepackage{fullpage}
\usepackage{amsmath,amssymb,amsfonts,amsthm}
\theoremstyle{definition}
\newtheorem{definition}{Definition}
\theoremstyle{lemma}
\newtheorem{lemma}{Lemma}
\newtheorem*{remark}{Remark}
\theoremstyle{theorem}
\newtheorem{theorem}{Theorem}
%\usepackage{thmtools}
%\declaretheorem[style=definition,qed=$\blacksquare$,numberwithin=chapter]{definition}

\usepackage[titletoc,toc,title]{appendix}

\usepackage{array} 
\usepackage{listings}
\usepackage{mathtools}
\usepackage{pdfpages}
\usepackage[textsize=footnotesize,color=green]{todonotes}
\usepackage{bm}
\usepackage{bbm}

%\usepackage{tikz}
\usepackage[normalem]{ulem}
\usepackage{hhline}

%% ====================================== alg package
\usepackage{algorithm}
\usepackage[noend]{algpseudocode}
\usepackage{algorithmicx}
\algblock{ParFor}{EndParFor}
% customising the new block
\algnewcommand\algorithmicparfor{\textbf{parfor}}
\algnewcommand\algorithmicpardo{\textbf{do}}
\algnewcommand\algorithmicendparfor{\textbf{end\ parfor}}
\algrenewtext{ParFor}[1]{\algorithmicparfor\ #1\ \algorithmicpardo}
\algrenewtext{EndParFor}{\algorithmicendparfor}
%% ====================================== end alg package

\usepackage{graphicx}
\usepackage{subfig}
\usepackage{color}

%% ====================================== graphics

\usepackage{pgfplots}
\usepackage{pgfplotstable}
\definecolor{markercolor}{RGB}{124.9, 255, 160.65}
\pgfplotsset{width=10cm,compat=1.3}
\pgfplotsset{
tick label style={font=\small},
label style={font=\small},
legend style={font=\small}
}

\usetikzlibrary{calc}

%%% START MACRO FOR ANNOTATION OF TRIANGLE WITH SLOPE %%%.
\newcommand{\logLogSlopeTriangle}[5]
{
    % #1. Relative offset in x direction.
    % #2. Width in x direction, so xA-xB.
    % #3. Relative offset in y direction.
    % #4. Slope d(y)/d(log10(x)).
    % #5. Plot options.

    \pgfplotsextra
    {
        \pgfkeysgetvalue{/pgfplots/xmin}{\xmin}
        \pgfkeysgetvalue{/pgfplots/xmax}{\xmax}
        \pgfkeysgetvalue{/pgfplots/ymin}{\ymin}
        \pgfkeysgetvalue{/pgfplots/ymax}{\ymax}

        % Calculate auxilliary quantities, in relative sense.
        \pgfmathsetmacro{\xArel}{#1}
        \pgfmathsetmacro{\yArel}{#3}
        \pgfmathsetmacro{\xBrel}{#1-#2}
        \pgfmathsetmacro{\yBrel}{\yArel}
        \pgfmathsetmacro{\xCrel}{\xArel}

        \pgfmathsetmacro{\lnxB}{\xmin*(1-(#1-#2))+\xmax*(#1-#2)} % in [xmin,xmax].
        \pgfmathsetmacro{\lnxA}{\xmin*(1-#1)+\xmax*#1} % in [xmin,xmax].
        \pgfmathsetmacro{\lnyA}{\ymin*(1-#3)+\ymax*#3} % in [ymin,ymax].
        \pgfmathsetmacro{\lnyC}{\lnyA+#4*(\lnxA-\lnxB)}
        \pgfmathsetmacro{\yCrel}{\lnyC-\ymin)/(\ymax-\ymin)} % THE IMPROVED EXPRESSION WITHOUT 'DIMENSION TOO LARGE' ERROR.

        % Define coordinates for \draw. MIND THE 'rel axis cs' as opposed to the 'axis cs'.
        \coordinate (A) at (rel axis cs:\xArel,\yArel);
        \coordinate (B) at (rel axis cs:\xBrel,\yBrel);
        \coordinate (C) at (rel axis cs:\xCrel,\yCrel);

        % Draw slope triangle.
        \draw[#5]   (A)-- node[pos=0.5,anchor=north] {1}
                    (B)-- 
                    (C)-- node[pos=0.5,anchor=west] {#4}
                    cycle;
    }
}
%%% END MACRO FOR ANNOTATION OF TRIANGLE WITH SLOPE %%%.

\newcommand{\logLogSlopeTriangleNeg}[5]
{
    % #1. Relative offset in x direction.
    % #2. Width in x direction, so xA-xB.
    % #3. Relative offset in y direction.
    % #4. Slope d(y)/d(log10(x)).
    % #5. Plot options.

    \pgfplotsextra
    {
        \pgfkeysgetvalue{/pgfplots/xmin}{\xmin}
        \pgfkeysgetvalue{/pgfplots/xmax}{\xmax}
        \pgfkeysgetvalue{/pgfplots/ymin}{\ymin}
        \pgfkeysgetvalue{/pgfplots/ymax}{\ymax}

        % Calculate auxilliary quantities, in relative sense.
        \pgfmathsetmacro{\xArel}{#1}
        \pgfmathsetmacro{\yArel}{#3}
        \pgfmathsetmacro{\xBrel}{#1-#2}
        \pgfmathsetmacro{\yBrel}{\yArel}
        \pgfmathsetmacro{\xCrel}{\xArel}

        \pgfmathsetmacro{\lnxB}{\xmin*(1-(#1-#2))+\xmax*(#1-#2)} % in [xmin,xmax].
        \pgfmathsetmacro{\lnxA}{\xmin*(1-#1)+\xmax*#1} % in [xmin,xmax].
        \pgfmathsetmacro{\lnyA}{\ymin*(1-#3)+\ymax*#3} % in [ymin,ymax].
        \pgfmathsetmacro{\lnyC}{\lnyA+#4*(\lnxA-\lnxB)}
        \pgfmathsetmacro{\yCrel}{\lnyC-\ymin)/(\ymax-\ymin)} % THE IMPROVED EXPRESSION WITHOUT 'DIMENSION TOO LARGE' ERROR.

        % Define coordinates for \draw. MIND THE 'rel axis cs' as opposed to the 'axis cs'.
        \coordinate (A) at (rel axis cs:\xArel,\yArel);
        \coordinate (B) at (rel axis cs:\xBrel,\yBrel);
        \coordinate (C) at (rel axis cs:\xCrel,\yCrel);

        % Draw slope triangle.
        \draw[#5]   (A)-- node[pos=.5,anchor=south] {1}
                    (B)-- 
                    (C)-- node[pos=0.5,anchor=west] {#4}
                    cycle;
    }
}
%%% END MACRO FOR ANNOTATION OF TRIANGLE WITH SLOPE %%%.

%%% START MACRO FOR ANNOTATION OF TRIANGLE WITH SLOPE %%%.
\newcommand{\logLogSlopeTriangleFlipNeg}[5]
{
    % #1. Relative offset in x direction.
    % #2. Width in x direction, so xA-xB.
    % #3. Relative offset in y direction.
    % #4. Slope d(y)/d(log10(x)).
    % #5. Plot options.

    \pgfplotsextra
    {
        \pgfkeysgetvalue{/pgfplots/xmin}{\xmin}
        \pgfkeysgetvalue{/pgfplots/xmax}{\xmax}
        \pgfkeysgetvalue{/pgfplots/ymin}{\ymin}
        \pgfkeysgetvalue{/pgfplots/ymax}{\ymax}

        % Calculate auxilliary quantities, in relative sense.
        %\pgfmathsetmacro{\xArel}{#1}
        %\pgfmathsetmacro{\yArel}{#3}
        \pgfmathsetmacro{\xBrel}{#1-#2}
        \pgfmathsetmacro{\yBrel}{#3}
        \pgfmathsetmacro{\xCrel}{#1}

        \pgfmathsetmacro{\lnxB}{\xmin*(1-(#1-#2))+\xmax*(#1-#2)} % in [xmin,xmax].
        \pgfmathsetmacro{\lnxA}{\xmin*(1-#1)+\xmax*#1} % in [xmin,xmax].
        \pgfmathsetmacro{\lnyA}{\ymin*(1-#3)+\ymax*#3} % in [ymin,ymax].
        \pgfmathsetmacro{\lnyC}{\lnyA+#4*(\lnxA-\lnxB)}
        \pgfmathsetmacro{\yCrel}{\lnyC-\ymin)/(\ymax-\ymin)} % THE IMPROVED EXPRESSION WITHOUT 'DIMENSION TOO LARGE' ERROR.

	\pgfmathsetmacro{\xArel}{\xBrel}
        \pgfmathsetmacro{\yArel}{\yCrel}

        % Define coordinates for \draw. MIND THE 'rel axis cs' as opposed to the 'axis cs'.
        \coordinate (A) at (rel axis cs:\xArel,\yArel);
        \coordinate (B) at (rel axis cs:\xBrel,\yBrel);
        \coordinate (C) at (rel axis cs:\xCrel,\yCrel);

        % Draw slope triangle.
        \draw[#5]   (A)-- node[pos=0.5,anchor=east] {#4}
                    (B)-- 
                    (C)-- node[pos=0.5,anchor=north] {1}
                    cycle;
    }
}
%%% END MACRO FOR ANNOTATION OF TRIANGLE WITH SLOPE %%%.


%%% START MACRO FOR ANNOTATION OF TRIANGLE WITH SLOPE %%%.
\newcommand{\logLogSlopeTriangleFlip}[5]
{
    % #1. Relative offset in x direction.
    % #2. Width in x direction, so xA-xB.
    % #3. Relative offset in y direction.
    % #4. Slope d(y)/d(log10(x)).
    % #5. Plot options.

    \pgfplotsextra
    {
        \pgfkeysgetvalue{/pgfplots/xmin}{\xmin}
        \pgfkeysgetvalue{/pgfplots/xmax}{\xmax}
        \pgfkeysgetvalue{/pgfplots/ymin}{\ymin}
        \pgfkeysgetvalue{/pgfplots/ymax}{\ymax}

        % Calculate auxilliary quantities, in relative sense.
        %\pgfmathsetmacro{\xArel}{#1}
        %\pgfmathsetmacro{\yArel}{#3}
        \pgfmathsetmacro{\xBrel}{#1-#2}
        \pgfmathsetmacro{\yBrel}{#3}
        \pgfmathsetmacro{\xCrel}{#1}

        \pgfmathsetmacro{\lnxB}{\xmin*(1-(#1-#2))+\xmax*(#1-#2)} % in [xmin,xmax].
        \pgfmathsetmacro{\lnxA}{\xmin*(1-#1)+\xmax*#1} % in [xmin,xmax].
        \pgfmathsetmacro{\lnyA}{\ymin*(1-#3)+\ymax*#3} % in [ymin,ymax].
        \pgfmathsetmacro{\lnyC}{\lnyA+#4*(\lnxA-\lnxB)}
        \pgfmathsetmacro{\yCrel}{\lnyC-\ymin)/(\ymax-\ymin)} % THE IMPROVED EXPRESSION WITHOUT 'DIMENSION TOO LARGE' ERROR.

	\pgfmathsetmacro{\xArel}{\xBrel}
        \pgfmathsetmacro{\yArel}{\yCrel}

        % Define coordinates for \draw. MIND THE 'rel axis cs' as opposed to the 'axis cs'.
        \coordinate (A) at (rel axis cs:\xArel,\yArel);
        \coordinate (B) at (rel axis cs:\xBrel,\yBrel);
        \coordinate (C) at (rel axis cs:\xCrel,\yCrel);

        % Draw slope triangle.
        \draw[#5]   (A)-- node[pos=0.5,anchor=east] {#4}
                    (B)-- 
                    (C)-- node[pos=0.5,anchor=south] {1}
                    cycle;
    }
}
%%% END MACRO FOR ANNOTATION OF TRIANGLE WITH SLOPE %%%.



\usepackage{stmaryrd}


\renewcommand{\topfraction}{0.85}
\renewcommand{\textfraction}{0.1}
\renewcommand{\floatpagefraction}{0.75}

\newcommand{\vect}[1]{\ensuremath\boldsymbol{#1}}
\newcommand{\tensor}[1]{\underline{\bm{#1}}}
\newcommand{\del}{\triangle}
\newcommand{\curl}{\grad \times}
\renewcommand{\div}{\grad \cdot}

\newcommand{\bbm}[1]{\mathbbm{#1}}
\newcommand{\bs}[1]{\boldsymbol{#1}}
\newcommand{\equaldef}{\stackrel{\mathrm{def}}{=}}

\newcommand{\td}[2]{\frac{{\rm d}#1}{{\rm d}{\rm #2}}}
\newcommand{\pd}[2]{\frac{\partial#1}{\partial#2}}
\newcommand{\pdd}[2]{\frac{\partial^2#1}{\partial#2^2}}
\newcommand{\pdn}[3]{\frac{\partial^{#3}#1}{\partial#2^{#3}}}
\newcommand{\mb}[1]{\mathbf{#1}}
\newcommand{\mbb}[1]{\mathbb{#1}}
\newcommand{\mc}[1]{\mathcal{#1}}
\newcommand{\snor}[1]{\left| #1 \right|}
\newcommand{\nor}[1]{\left\| #1 \right\|}
\newcommand{\LRp}[1]{\left( #1 \right)}
\newcommand{\LRs}[1]{\left[ #1 \right]}
\newcommand{\LRa}[1]{\left\langle #1 \right\rangle}
\newcommand{\LRb}[1]{\left| #1 \right|}
\newcommand{\LRc}[1]{\left\{ #1 \right\}}
\newcommand{\LRceil}[1]{\left\lceil #1 \right\rceil}
\newcommand{\LRl}[1]{\left. #1 \right|}

%\newcommand{\cond}[1]{\kappa\LRp{#1}}
\newcommand{\cond}[2]{\nor{#1}_{#2}\nor{{#1}^{-1}}_{#2}}


\newcommand{\Grad} {\ensuremath{\nabla}}
\newcommand{\Div} {\ensuremath{\nabla\cdot}}
\newcommand{\jump}[1] {\ensuremath{\llbracket#1\rrbracket}}
\newcommand{\avg}[1] {\ensuremath{\LRc{\!\{#1\}\!}}}

\newcommand{\Oh}{{\Omega_h}}
\renewcommand{\L}{L^2\LRp{\Omega}}
\newcommand{\LK}{L^2\LRp{D^k}}
\newcommand{\LdK}{L^2\LRp{\partial D^k}}
\newcommand{\Dhat}{\widehat{D}}
\newcommand{\Lhat}{L^2\LRp{\Dhat}}

\newcommand{\eval}[2][\right]{\relax
  \ifx#1\right\relax \left.\fi#2#1\rvert}

\def\etal{{\it et al.~}}


\newcommand{\note}[1]{{\color{blue}{#1}}}


\newcommand{\LinfDk}{L^{\infty}\LRp{D^k}}

\newcommand{\diag}[1]{{\rm diag}\LRp{#1}}

\newcommand{\Ksub}{K_{\rm sub}}

\newcolumntype{C}[1]{>{\centering\let\newline\\\arraybackslash\hspace{0pt}}m{#1}}

%% d in integrand
\newcommand*\diff[1]{\mathop{}\!{\mathrm{d}#1}}


\makeatletter
\renewcommand\d[1]{\mspace{6mu}\mathrm{d}#1\@ifnextchar\d{\mspace{-3mu}}{}}
\makeatother

\date{}
\author{Jesse Chan}
\title{On discrete entropy conservation for a more general class of discontinuous Galerkin methods}

\begin{document}

\maketitle

\begin{abstract}
High order methods based on diagonal-norm summation by parts operators can be shown to semi-discretely conserve entropy for nonlinear systems of hyperbolic PDEs \cite{fisher2013high,carpenter2014entropy}.  These include discontinuous Galerkin spectral element methods (DG-SEM) \cite{gassner2016split,gassner2016well,wintermeyer2017entropy, chen2017entropy} based on mass lumping and nodal collocation.  In this work, we describe how to extend discretely entropy conservative schemes for a more general class of DG methods using quadrature-based projections and global DG differentiation operators.  The construct of these methods is independent of the choice of basis and accuracy of quadrature rule used, and includes DG methods involving generalized SBP operators without boundary nodes and certain dense-norm SBP operators.  Numerical experiments confirm the stability and high order accuracy of the proposed methods for the Burgers, shallow water, and compressible Euler equations.  
\end{abstract}

\tableofcontents

\section{Introduction}

\note{High order methods are good boilerplate}

\note{High order methods are not stable for nonlinear problems.  Stability of discretizations is a hot topic: artificial viscosity, polynomial de-aliasing,  \cite{karniadakis2013spectral}}

\cite{tadmor1987numerical, tadmor2003entropy}

Hughes entropy variables \cite{hughes1986new}

\note{Specify that entropy stability has been shown for mass lumped collocation DG discretizations.  Requirements are a diagonal norm SBP operators (or diagonal mass matrix) and collocated differentiation. }  \cite{gassner2016split,gassner2016well,wintermeyer2017entropy}  
\note{The construction of such operators on non-tensor product elements (such as simplices or pyramids) is non-trivial.} \cite{hicken2016multidimensional, chen2017entropy}.  \note{Constructing such operators for non-polynomial spaces (such as splines) is also difficult within the current framework.}

While DG-SEM based on flux differencing is entropy conservative, this is tied to a specific choice of basis and quadrature.  Furthermore, GLL quadratures and diagonal-norm SBP operators do not naturally generalize to triangles or tetrahedra \cite{helenbrook2009existence}.  Diagonal-norm SBP operators can be constructed for triangles and tetrahedra \cite{chin1999higher, cohen2001higher, hicken2016multidimensional, chen2017entropy}; however, the number of nodal points for such operators is greater than the dimension of the natural polynomial approximation space.  Furthermore, appropriate point sets have only been constructed for $N \leq 4$ in three dimensions \cite{zhebel2014comparison}.  

\section{Entropy conservative diagonal norm collocation DG methods}

\note{Specify what we mean by diagonal norm collocation DG method.}

\subsection{Entropy stability for systems of hyperbolic PDEs}

We consider systems of nonlinear conservation laws in $\mathbb{R}^d$ with $n$ variables
\begin{align}
\pd{\bm{u}}{t} + \sum_{k=1}^d \pd{\bm{f_k(\bm{u})}}{\bm{x}_k} &= 0, \qquad \bm{u}(\bm{x},t) = (u_1(\bm{x},t),\ldots,u_n(\bm{x},t)).
\label{eq:pde}
\end{align}
where the fluxes $f_k(\bm{u})$ are smooth functions of the vector of conservative variables $\bm{u}(\bm{x},t)$.
The Jacobian matrix $\bm{A}_k(\bm{u})$ is defined entrywise as
\[
\LRp{\bm{A}_k(\bm{u})}_{ij} = \LRp{\pd{\bm{f}_k(\bm{u})}{\bm{u}_j}}_i.
\]
We are interested in systems for which there exists a convex entropy function $U(\bm{u})$ such that  
\begin{equation}
U''(\bm{u})\bm{A}_k(\bm{u}) = \LRp{U''(\bm{u}) \bm{A}_k(\bm{u})}^T.
\label{eq:entropysym}
\end{equation}
For systems with convex entropy functions, one can define entropy variables $\bm{v} = U'(\bm{u})$.  The convexity of the $U(\bm{u})$ guarantees that the mapping between conservative and entropy variables is invertible.  

It can be shown (see, for example, \cite{mock1980systems}) that (\ref{eq:entropysym}) is equivalent to the existence of an entropy flux functions $F_k(\bm{u})$ such that
\[
\bm{v}^T \pd{\bm{f}_k}{\bm{u}} = \pd{F_k(\bm{u})}{\bm{u}}^T.
\]
The flux function and entropy flux function are further related through the so-called entropy potential $\psi_k$ 
\[
\psi_k(\bm{v}) = \bm{v}^T\bm{f}_k(\bm{u}(\bm{v})) - F_k(\bm{u}(\bm{v})), \qquad \psi_k'(\bm{v}) = \bm{f}_k(\bm{u}(\bm{v})).
\]

When $\bm{u}$ is smooth, multiplying (\ref{eq:pde}) on the left by $\bm{v}^T = U'(\bm{u})^T$, applying the definition of the entropy flux and using the chain rule yields the conservation of entropy
\[
\pd{U(\bm{u})}{t} + \sum_{k=1}^d \pd{F_k(\bm{u})}{\bm{x}_k} = 0.
\]
More generally, it can be shown that physically relevant solutions of (\ref{eq:pde}) (defined as the limiting solution for an appropriately defined vanishing viscosity) satisfy the entropy inequality
\[
\pd{U(\bm{u})}{t} + \sum_{k=1}^d \pd{F_k(\bm{u})}{\bm{x}_k} \leq 0.  
\]
%Integration over $\mathbb{R}^d$ then shows that the entropy of the solution decreases in time.  
When $d = 1$, integrating the entropy inequality over $[-1,1]$ and using the definition of the entropy potential yields
\begin{equation}
\int_{-1}^1 \pd{U(\bm{u})}{t} + \left.\LRp{\bm{v}^T\bm{f}_k(\bm{u}(\bm{v}))-\psi(\bm{v})}\right|_{-1}^1 \leq 0.  
\label{eq:consentropy}
\end{equation}

%\note{Wave equation example}


\subsection{Entropy conservative collocation methods based on SBP operators}

Let $\widehat{x}_i \in [-1,1]$ denote the $(N+1)$ point Gauss-Lobatto-Legendre (GLL) quadrature rule with positive weights $w_1,\ldots,w_{N+1}$.  Let $\ell_i(x)$ denote the nodal Lagrange basis defined at GLL points, and define the differentiation matrix $\bm{D}$ be defined as 
\[
\bm{D}_{ij} = \pd{\ell_j(x_i)}{x}.
\]
The matrix $\bm{D}$ maps from nodal values of a polynomial to nodal values of its derivative.  We also define the mass matrix $\bm{M}$
\[
\bm{M}_{ij} = \int_{-1}^1 \ell_i(x)\ell_j(x)\diff{x} \approx \delta_{ij} w_i,
\]
where the approximation results evaluating the integral using GLL quadrature.  For this choice of quadrature, the Kronecker property $\ell_j(x_i) = \delta_{ij}$ gives that $\bm{M}$ is a diagonal matrix.  

The mass and derivative matrices may be used to define the diagonal norm SBP operator $\bm{S} = \bm{M}\bm{D}$, with the summation-by-parts property 
\[
\bm{S} = \bm{B} - \bm{S}^T, \qquad \bm{B}_{ij} = \begin{cases}
1 &i = j = 1 \\
1 &i = j = N+1 \\
0 &\text{otherwise}
\end{cases}.
\]
This property is simply a restatement of integration by parts, using that the derivative of $\ell_j(x) \in P^{N-1}$ and that GLL quadrature is exact for polynomials of degree $2N-1$.  

We seek a polynomial approximation to the solution $\bm{u}(x)$ of (\ref{eq:pde}).  A DG formulation in the ``strong form'' \cite{hesthaven2007nodal} is given as
\[
\int_{-1}^1\LRp{ \pd{\bm{u}}{t} + \pd{\bm{f}(\bm{u})}{x}}\bm{w} + \left.\LRp{\bm{f}^* - \bm{f}(\bm{u})} \bm{w}\right|_{-1}^1 = 0, \qquad \forall \bm{w}\in V_h.
\]
Here, $\bm{f}^*$ is a numerical flux used for the weak imposition of boundary or continuity conditions.  

We derive a DG-SEM approximation by evaluating the above integrals using GLL quadrature.  We furthermore assume that the solution is approximated using a nodal basis at GLL points 
\[
\bm{u}(\bm{x},t) = \sum_{j=1}^{N+1} \bm{u}_j(t)\ell_j(x).  
\]
The resulting semi-discrete system yields a nodal collocation scheme for $\bm{u}_i$ 
\[
\pd{\bm{u}_i}{t} + \sum_{j=1}^{N+1}\LRp{\LRp{\bm{D}\otimes \bm{I}}_{ij}\bm{f}(\bm{u}_j) +\LRp{\bm{M}^{-1}\bm{B}}_{ij}(\bm{f}_i^* - \bm{f}(\bm{u}_i))} = 0.
\]

This formulation conserves the quantities $\bm{u}_i$, but (except in specific cases) does not satisfy a discrete analogue of the statement of entropy conservation (\ref{eq:consentropy}).  Entropy conservative schemes may be constructed based on the SBP property and a ``flux differencing'' approach \cite{gassner2016split} involving a definition of entropy conservative/stable fluxes given by Tadmor \cite{tadmor1987numerical, tadmor2003entropy}.  
\begin{definition}
Let $\bm{f}_S(\bm{u}_L,\bm{u}_R)$ be a bivariate function such that it is symmetric and consistent with the flux function $\bm{f}(\bm{u})$
\[
\bm{f}_S(\bm{u}_L,\bm{u}_R) = \bm{f}_S(\bm{u}_R,\bm{u}_L), \qquad \bm{f}_S(\bm{u},\bm{u}) = \bm{f}(\bm{u})
\]
The numerical flux $\bm{f}_S(\bm{u}_L, \bm{u}_R)$ is entropy conservative if, for entropy variables $\bm{v}_L = \bm{v}(\bm{u}_L), \bm{v}_R = \bm{v}(\bm{u}_R)$
\[
\LRp{\bm{v}_L - \bm{v}_R}^T \bm{f}_S(\bm{u}_L,\bm{u}_R) = (\psi_L - \psi_R), \qquad \psi_L = \psi(\bm{v}(\bm{u}_L)), \quad \psi_R = \psi(\bm{v}(\bm{u}_R)).  
\]
Similarly, a flux $\bm{f}_S(\bm{u}_L, \bm{u}_R)$ is referred to as entropy stable if $\LRp{\bm{v}_L - \bm{v}_R}^T \bm{f}_S(\bm{u}_L,\bm{u}_R) \leq (\psi_L - \psi_R)$.
\label{def:tadmor}
\end{definition}
An entropy conservative nodal collocation method can then be derived by modifying the flux derivative as follows
\begin{equation}
\pd{\bm{u}_i}{t} + \sum_{j=1}^{N+1}\LRp{\LRp{2\bm{D}\otimes \bm{I}}_{ij}\bm{f}_S(\bm{u}_i,\bm{u}_j) +\LRp{\bm{M}^{-1}\bm{B}}_{ij}(\bm{f}_i^* - \bm{f}(\bm{u}_i))}= 0.
\label{eq:dgsem}
\end{equation}
The replacement of the derivative of the flux function with the flux differencing allows one to prove that (\ref{eq:dgsem}) conserves a discrete entropy \cite{gassner2017br1,chen2017entropy}.  We reproduce these proofs here, as we will refer to them in following sections.  
\begin{theorem}
The collocation method (\ref{eq:dgsem}) is discretely entropy conservative in the sense that
\[
\bm{1}^T\LRp{\bm{M}\pd{U(\bm{u})}{t} + \bm{B}\LRp{\bm{v}^T\bm{f}^* - \bm{\psi}}} = 0,
\]
where $\bm{1}$ denotes the vector of ones and $\bm{\psi}_i = \psi(\bm{v}_i)$ is the entropy potential in terms of the nodal value of $\bm{u}_i$.  
\label{thm:dgsem}
\end{theorem}
\begin{proof}
Let $\bm{v}_i = \bm{v}(\bm{u}_i)$ denote the entropy variables, defined in terms of the nodal values $\bm{u}_i$.  We multiply (\ref{eq:dgsem}) by $\LRp{\bm{M}\bm{v}}^T$.  
Using the fact that $\bm{M}$ is diagonal, the time derivative term can be manipulated to 
\[
\bm{v}^T\bm{M}\pd{\bm{u}}{t} = \sum_{i=1}^{N+1} \bm{M}_{ii} \bm{v}_i^T \pd{\bm{u}_i}{t} = \sum_{i=1}^{N+1} \bm{M}_{ii} \pd{S_i}{t} = \bm{1}^T\bm{M}\pd{S_i}{t},
\]
where we have used continuity in time to arrive at the pointwise relation 
\[
\bm{v}^T \pd{\bm{u}}{t} = \pd{U(\bm{u})}{\bm{u}}^T \pd{\bm{u}}{t} = \pd{U(\bm{u})}{t}.
\]

The spatial derivative term can be manipulated using the summation by parts property $\bm{M}\bm{D} = \bm{S} = \bm{B}-\bm{S}^T$
\begin{align}
\sum_{i,j=1}^{N+1} \bm{v}_i^T 2 \bm{M}\bm{D}_{ij}\bm{f}_S(\bm{u}_i,\bm{u}_j) &= \sum_{i,j=1}^{N+1} \bm{v}_i^T (\bm{S} + \bm{B} - \bm{S}^T)_{ij}\bm{f}_S(\bm{u}_i,\bm{u}_j) \nonumber\\
&= \sum_{i,j=1}^{N+1} \bm{S}_{ij}\bm{v}_i^T \bm{f}_S(\bm{u}_i,\bm{u}_j) - \bm{S}_{ji}\bm{v}_i^T \bm{f}_S(\bm{u}_i,\bm{u}_j) + \bm{B}_{ij}\bm{v}_i^T \bm{f}_S(\bm{u}_i,\bm{u}_j). \label{eq:sbpentropy}
\end{align}
Rearranging indices and using symmetry of $\bm{f}_S(\bm{u}_i,\bm{u}_j)$ then yields that
\begin{align*}
\sum_{i,j=1}^{N+1} \bm{S}_{ij}\bm{v}_i^T \bm{f}_S(\bm{u}_i,\bm{u}_j) - \bm{S}_{ji}\bm{v}_i^T \bm{f}_S(\bm{u}_i,\bm{u}_j) &= \sum_{i,j=1}^{N+1} \bm{S}_{ij} \LRp{\bm{v}_i - \bm{v}_j}^T\bm{f}_S(\bm{u}_i,\bm{u}_j)\\
&= \sum_{i,j=1}^{N+1} \bm{S}_{ij} \LRp{\bm{\psi}_i-\bm{\psi}_j} = -\bm{1}^T \bm{S}\bm{\psi} = -\bm{1}^T\bm{B}\bm{\psi},
\end{align*}
where we have used that $\bm{S}\bm{1} = 0$ and the summation by parts property in the last line.  

Using that $\bm{B}$ is diagonal, combining the latter term in (\ref{eq:sbpentropy}) with boundary terms involving the numerical flux yields
\[
\sum_{i,j = 1}^{N+1} \bm{B}_{ij}\bm{v}_i^T \bm{f}_S(\bm{u}_i,\bm{u}_j) + \bm{v}_i^T\bm{B}_{ij}\LRp{\bm{f}_i^* - \bm{f}(\bm{u}_i)} = \bm{1}^T\bm{B} \LRp{\bm{v}^T \bm{f}^*},
\]
where we have used the consistency of $\bm{f}_S(\bm{u},\bm{u}) = \bm{f}(\bm{u})$.  Combining everything results in a discrete statement of the conservation of entropy (\ref{eq:consentropy}).   
%Combining everything results in a discrete statement of the conservation of entropy (\ref{eq:consentropy}) 
%\[
%\bm{1}^T\LRp{\bm{M}\pd{U(\bm{u})}{t} + \bm{B}\LRp{\bm{v}^T\bm{f}^* - \bm{\psi}}} = 0.
%\]
\end{proof}

\begin{remark}
The proof of entropy conservation can be extended beyond a single domain by specifying appropriate simultaneous approximation terms (SBP-SAT).  This can be done by taking $\bm{f}_i^*$ to be the entropy conservative flux $f_S(u_L,u_R)$ at the interface between two elements \cite{carpenter2014entropy,gassner2017br1,chen2017entropy}.  Similarly, employing an entropy stable flux at element interfaces results in an entropy stable method which satisfies a global entropy inequality.  
\end{remark}


\section{Entropy conservative general DG methods}

In this section, we focus on constructing entropy conservative DG schemes with arbitrary bases and quadrature rules.  These include dense norm SBP operators and generalized SBP (GSBP) operators which do not contain boundary points, which introduce additional difficulties in constructing entropy conservative DG schemes.  These difficulties present themselves primarily in three areas:
\begin{enumerate}
\item It is unclear how to generalize the flux differencing approach to arbitrary choices of quadrature.  
\item The proof of entropy stability relies on a diagonal norm mass matrix, which commutes with pointwise multiplication by the (nodal values of) the entropy variables.  
\item Entropy stable formulations based on dense norm or GSBP operators require non-trivial SBP-SAT terms \cite{ranocha2017extended}.  These can be derived by hand for simple problems such as Burgers' equation, but can be difficult to determine for more complex systems of nonlinear PDEs.  
\end{enumerate}

We simultaneously address all three areas of difficulty by introducing a generalization of the entropy conservative DG-SEM scheme (\ref{eq:dgsem}) to arbitrary quadratures and choices of basis.  Additionally, we prove that the generalized scheme is entropy conservative (stable) on multiple elements in the following sections.  

%We will get around these difficulties for dense-norm DG methods by satisfying three requirements for entropy stability
%\begin{itemize}
%\item $\bm{D}\bm{1} = 0$ for use of the flux formulation.  
%\item For some norm matrix $\bm{W}$, $\bm{W}\bm{D} = \bm{B}-\bm{D}^T\bm{W}$ where $\bm{B}$ is a suitably defined boundary operator.
%\item Contraction with the \textit{projection} of the entropy variables.  
%\end{itemize}

\subsection{Notation}

We assume that the domain $\Omega \in \mathbb{R}^d$ is exactly triangulated by a mesh which consists of the union of non-overlapping elements $D^k$.  We further assume that each element $D^k$ is the image of a reference element $\widehat{D}$ under the local elemental mapping 
\[
\bm{x}^k = \bm{\Phi}^k \widehat{\bm{x}},
\]
where $\bm{x}^k$ denote physical coordinates on $D^k$ and $\widehat{\bm{x}}$ denote coordinates on the reference element.  Finally, we identify $J$ as the determinant of the Jacobian of $\bm{\Phi}^k$, and refer to it as the Jacobian for the remainder of this work.  

We will approximate solution components over each element $D^k$ from an approximation space $V_h\LRp{D^k}$, which we define as the composition of the mapping $\bm{\Phi}^k$ and a reference approximation space $V_h\LRp{\widehat{D}}$
\[
V_h\LRp{D^k} = \bm{\Phi}^k \circ V_h\LRp{\widehat{D}}.
\]
The global approximation space $V_h\LRp{\Omega}$ is then defined as the direct sum of local approximation spaces
\[
V_h\LRp{\Omega} = \bigoplus_{D^k}V_h\LRp{D^k}.  
\]
We assume that $V_h(D^k) \subset L^2(D^k)$; in other words, that all elements of each local approximation space are square integrable.  We also define the local $L^2$ projection operator $\Pi_N: L^2(D^k)\rightarrow V_h(D^k)$ such that
\[
\LRp{\Pi_N u,v}_{D^k} = \LRp{u,v}_{D^k}, \qquad \forall v\in V_h\LRp{D^k}.  
\]
Because the global approximation space $V_h\LRp{\Omega}$ is the direct sum of independent local approximation spaces, the global $L^2$ projection is simply the composition of local $L^2$ projections.  

%\note{We define the quadrature-based $L^2$ projection $\Pi_N$ as the $L^2$ projection where integrals are evaluated using some (inexact) quadrature.  We also define the discrete approximation space $V_h$, and assume that the quadrature is sufficiently accurate such that integration by parts holds for any two elements of $V_h$.  Finally, we assume that the discrete derivative (to be defined) maps from $V_h$ to $V_h$.  }

Finally, we define the jump and average of discontinuous functions across element interfaces.  Let $D^{k,+}$ denote the neighboring element of across a face $f$ of $D^k$, and let $u^+,u^-$ denote the values of $u$ on $D^{k,+}$ and $D^k$, respectively.  The jump of $u$ across $f$ is then defined as
\[
\jump{u} = u^+ - u^-, \qquad \avg{u} = \frac{u^+ + u^-}{2}.
\]
The jump and average of vector fields are defined component-wise using the jumps and averages of components.  

%\note{Specify assumption that differentiation operators $\bm{D}:V_h\rightarrow V_h$}

\subsection{A continuous interpretation of flux differencing}

In this section, we offer a continuous interpretation of flux differencing \cite{fisher2013high,gassner2017br1,chen2017entropy}, which also encompasses the split form methodology proposed in \cite{gassner2016split}.  This interpretation will guide the construction of entropy conservative DG schemes.  

Assume that $\bm{f}_S(\bm{u}_L,\bm{u}_R)$ is the symmetric and consistent two-point flux defined in (\ref{def:tadmor}).  Consider the bilinear function
\[
\bm{f}_S\LRp{\bm{u}(\bm{x}),\bm{u}(\bm{y})}, \qquad \bm{x}, \bm{y} \in \mathbb{R}^d.  
\]
Flux differencing can be interpreted as differentiating $\bm{f}_S\LRp{\bm{u}(\bm{x}),\bm{u}(\bm{y})}$ along one coordinate $\bm{x}_k$, then evaluating the resulting derivative with $\bm{y}=\bm{x}$.  Since the two-point flux is consistent, this results in the following approximation of the flux derivative
\begin{equation}
\pd{\bm{f}(\bm{u}(\bm{x}))}{\bm{x}_k} \Longrightarrow 2\left.\pd{\bm{f}_S\LRp{\bm{u}(\bm{x}),\bm{u}(\bm{y})}}{\bm{x}_k}\right|_{\bm{y}=\bm{x}}.
\label{eq:fluxdiff}
\end{equation}

\subsubsection{Illustration using Burgers' equation}

We note that flux differencing allows for the recovery of split formulations.  We illustrate this using the one-dimensional Burgers equation, which is given in conservative form as
\[
\pd{u}{t} + \pd{f(u)}{x} = 0, \qquad f(u) = \frac{1}{2}u^2.
\]
It is well known that DG formulations based on the split form
\[
\pd{u}{t} + \frac{1}{3}\LRp{\pd{u^2}{x} + u\pd{u}{x}} = 0
\]
can be shown to be discretely energy stable under appropriately imposed boundary conditions \cite{gassner2013skew, ranocha2017extended}.  This form may be recovered using flux differencing by choosing the two-point flux
\[
f_S(u_L,u_R) = \frac{1}{6}(u_L^2 + u_Lu_R + u_R^2).
\]
Then, applying (\ref{eq:fluxdiff}) yields
\[
2\left.\pd{f_S(u(x),u(y))}{x}\right|_{y=x} = \frac{1}{3}\left.\pd{\LRp{u(x)^2 + u(x)u(y) + u(y)^2}}{x}\right|_{y=x} = \frac{1}{3}\LRp{\pd{u^2}{x} + u\pd{u}{x}},
\]
due to the fact that $\pd{u(y)^2}{x} = 0$.  

Both the conservative and split forms of Burgers' equation are equivalent at the continuous level (assuming continuity of $u$ and exact differentiation); however, this is no longer true at the discrete level.  Suppose now that the solution is approximated by $u \in V_h(\Omega)$ on a single element.  Since $f(u) = u^2/2 \not\in V_h(\Omega)$, it is not possible to differentiate $f(u)$ exactly without prior knowledge of the form of the nonlinearity.  Instead, it is more common to first project $f(u)$ to $V_h$, then differentiate, such that the discrete scheme is equivalent to solving for $u \in V_h(\Omega)$ such that
\[
\int_{-1}^1 \LRp{\pd{u}{t} + \pd{\Pi_N f(u)}{x}}w = 0, \qquad \forall w \in V_h(\Omega).  
\]
This scheme is energy stable only for projections which are computed using a quadrature of degree $3N-1$, which is sufficiently accurate to integrate the nonlinear DG weak form exactly.  

In comparison, when applying (\ref{eq:fluxdiff}) using $\pd{\Pi_N (\cdot)}{x}$, the resulting right hand side for Burgers' equation becomes 
\[
\int_{-1}^1\LRp{\pd{u}{t} + \frac{1}{3}\LRp{\pd{\LRp{\Pi_N u^2}}{x} + \Pi_N \LRp{u\pd{u}{x}}}}w = 0, \qquad \forall w \in V_h(\Omega).  
\]
Using the fact integration by parts holds when considering two elements of $V_h$, and that $\LRp{\Pi_N u^2, v} = \LRp{u^2,v}$ for any $v \in V_h$, it can be shown that the volume terms cancel out for $v = u$, leaving only boundary terms (which is a prerequisite of energy/entropy stability).  

\note{De-aliasing by increasing the degree of quadrature hits a ``bottleneck'': in the end, you are only differentiating a polynomial of degree $N$ \cite{kirby2003aliasing}.}

We note that the recovery of the split-form of Burgers' equation in the above example relies only on the property that 
\[
\pd{f(u(y))}{x} = f(u(y)) \pd{1}{x} = 0.
\]
As such, replacing $\pd{}{x}$ by any differential operator $D$ such that $D 1 = 0$ will recover a similar formulation.

\subsubsection{Recovering the algebraic form of flux differencing}

This approach yields the following evaluation (in Matlab for lack of better notation): let 
\[
\verb|fS = @(uL,uR) uL.^2+uL.*uR+uR.^2|
\]
be the flux function.  Let \verb+uq = Vq * u+ denote the interpolation of $u$ to quadrature points, and let \verb+Dx+ denote the differentiation matrix.  Define \verb+Pq = M\(Vq' * diag(wq))+, where \verb+wq+ are quadrature weights, and let \verb+[ux uy] = meshgrid(uq)+.  Then, assuming element-constant Jacobians, the split form can be evaluated via
\[
\verb+Pq * diag((Vq*Dx*Pq) * fS(ux, uy)) = Pq * sum((Vq* Dx* Pq) .* fS(ux, uy), 2).+
\]
The latter expression can be performed by computing \verb+fS(ux,uy)+ on the fly, avoiding the need to store it as a matrix.  It can also be generalized to handle non-square matrices in a very similar way.  

When using nodal collocation using either GLL or GQ points, \verb+Vq,Pq+ are the identity matrix and we recover the algebraic form of flux differencing described in \cite{gassner2016split}.

\note{Introduce definition of $\bm{f}(\bm{u}_q)$ as $f(\bm{u}_q)$.}

\subsection{Global DG differentiation operators}

The construction of entropy conservative DG methods 

Global DG methods \cite{di2011mathematical}

We adapt the construction of local differentiation operators by Chen and Shu \cite{chen2017entropy} to construct a global DG differentiation operator
\begin{equation}
\LRp{D^x_h u,v} = \sum_k \LRp{\pd{ \Pi_N u}{x},v}_{D^k} + \frac{1}{2}{\LRa{{u_P - \Pi_N u_M}, v\bm{n}_x}_{\partial D^k} + \frac{1}{2}\LRa{{u_M - \Pi_Nu_M },\Pi_Nv \bm{n}_x}_{\partial D^k}}.
\label{eq:dgd}
\end{equation}
Integrating by parts then yields the skew-symmetric form of $\bm{D}^x_h$
\[
\LRp{D^x_h u,v} =\frac{1}{2} \sum_k \LRp{\pd{ \Pi_N u}{x},v}_{D^k} - \LRp{u,\pd{\Pi_N v}{x}}_{D^k} + \LRa{\LRp{u_P - \Pi_N u_M}\bm{n}_x, v} + \LRa{{u_M }\bm{n}_x,\Pi_Nv}.
\]
Taking $v = u_M$ and cancelling $u_P u_M \bm{n}_x$ terms shows that the differentiation operator has the global SBP property and is skew-symmetric  w.r.t.\ a discrete $L^2$ inner product.  The coupling requires only traces of neighboring elements.  

It's currently unclear how to distinguish whether different operators are high order accurate.  

\begin{remark}
It is possible to define a simpler global DG differentiation operator based on the composition of a weak polynomial differentiation operator and the $L^2$ projection.  Let $\bm{D}^x_h: V_h\rightarrow V_h$ be defined as
\[
\sum_k \LRp{D^x_h u,v}_{D^k} = \sum_k \LRp{\pd{u}{x},v}_{D^k} + \frac{1}{2}\LRa{\jump{u},v\bm{n}_x}_{\partial D^k}, \qquad \forall v \in V_h.
\]
Then, it is straightforward to show that $D^x_h \Pi_N$ possesses a global SBP property with respect to the $L^2$ norm, such that
\[
\LRp{D^x_h\Pi_N u, v}_{\Omega} = \LRa{\Pi_N u,v}_{\partial \Omega} - \LRp{D^x_h\Pi_N v, u}_{\Omega}, \qquad \forall u,v\in \L.
\]
Numerical results using this global differentiation operator result in slightly more accurate results than when using the DG operator (\ref{eq:dgd}).  Unfortunately, compared to  (\ref{eq:dgd}), this construction results in a wide stencil involving all degrees of freedom on neighboring elements due to the presence of $\jump{u}$ and the $L^2$ projection.  
\end{remark}

\subsection{Discretely entropy conservative general DG methods}

We introduce 
\[
\int_{D^k}\pd{\bm{u}}{t} \bm{w} + \left.\LRp{D^x_h\bm{f}_S\LRp{\bm{u}(\Pi_N\bm{v}(\bm{x})),\bm{u}(\Pi_N\bm{v}(\bm{y}))}}\right|_{\bm{y}=\bm{x}}\bm{w} = 0, \qquad \forall \bm{w}\in V_h,
\]
where the global operator $D^x_h$ is understood to be applied to each component of $\bm{f}_S\LRp{\bm{u}(x),\bm{u}(y)}$.  

Global DG operators reveal how to construct SBP-SAT terms for any collocation-type DG methods (where $L^2$ projection and interpolation are equivalent).  However, for quadrature-based DG methods, it is not possible to test with the entropy variables directly because the test space is polynomial.  However, we may test with the projections of the entropy variables.  On the left hand side, we may use the fact that, under a method of lines discretization, 
\[
\bm{u}(x,t) = \sum_{j=1}^{N_p} \bm{u}_j(t) \phi_j(x), \qquad \pd{\bm{u}}{t} \in P^N.
\]
Thus, testing with $\Pi_N \bm{v}$ over each element $D^k$ yields  
\[
\int_{D^k} \pd{\bm{u}}{t}\Pi_N \bm{v} = \int_{D^k} \pd{\bm{u}}{t}\bm{v} = \int_{D^k} \pd{U(\bm{u})}{t}.  
\]
In matrix notation, this can be shown by multiplying with $\LRp{\bm{P}_q\bm{v}_q}^T$ 
\[
\LRp{\bm{P}_q\bm{v}_q}^T\bm{M}\pd{\bm{u}}{t} = \bm{v}_q^T\bm{W}\bm{V}_q\bm{M}^{-1}\bm{M}\pd{\bm{u}}{t} = \bm{1}^T \bm{W} {\rm diag}(\bm{v}_q) \pd{\bm{u}_q}{t} = \bm{1}^T \bm{W} \pd{U(\bm{u}_q)}{t},
\]
where we have assumed continuity in time, and have used that $\bm{W}$ is diagonal.  Summing contributions from each element $D^k$ yields the time derivative of the entropy over the entire domain $\Omega$. 

%which yields a pointwise relationship which can be manipulated into the time derivative of entropy.  

For collocation DG methods, interpolation and quadrature-based $L^2$ projection are identical.  However, when using a quadrature rule with $N_q > N_p$, this is not the case.  The solution is to evaluate the flux with $u(\Pi_N v)$ - in other words, evaluate the flux variables in terms of the projections of the entropy variables.  

Not easy to deal with SBP-SAT terms for more complex systems \cite{ranocha2017extended,ranocha2017comparison}.

We summarize this analysis in the following theorem:
\begin{theorem}
\note{Blah is entropy conservative.  Finish up proof + statement of entropy conservation.}
\end{theorem}


\begin{remark}
The analysis in this section has focused on the construction of entropy conservative schemes.  However, for problems with shocks, entropy is not conserved and should be dissipated away.  To this end, one can construct entropy stable schemes by adding additional stabilization terms (for example, by adding Lax-Friedrichs penalization or matrix dissipation terms \cite{chandrashekar2013kinetic,winters2017uniquely}).  
\end{remark}

\subsubsection{On implementation}

Let $V_h^f$ denote the trace space of the approximation space $V_h$ 
\[
V_h^f = \LRc{ \left.u\right|_{\partial D^k}, \quad u \in V_h}.  
\]
We introduce the lifting operator ${L}: V_h^f\rightarrow V_h$.  For $u_f \in V^f_h$, ${L}$ is defined as 
\[
\LRp{{L} u_f,v}_{D^k} = \LRa{u_f,v}_{\partial D^k}, \qquad \forall v\in V_h.
\]

\subsubsection{Example: Burgers equation}

We illustrate the application of the global flux differencing approach to Burgers' equation using the energy conserving flux
\[
f_S(u_L,u_R) = \frac{1}{6}\LRp{u_L^2 + u_Lu_R u_R^2}.  
\]
We will derive 
\[
\LRp{\left.\bm{D}^x_h f_S(u(x),u(y))\right|_{y=x}, w} = 
\]

We note that the global flux differencing approach with the DG operator (\ref{eq:dgd}) recovers the formulation and correction terms introduced in \cite{ranocha2017extended}.  

\subsection{Conservation of secondary properties}

It holds but with respect to specifically defined fluxes.  

\section{Numerical experiments}

\note{We illustrate the entropy conservation and accuracy of the proposed schemes for three equations.  These gradually increase in complexity.  }

\subsection{Burgers}

We begin by computing $L^2$ errors and rates of convergence for smooth solutions of Burgers' equation.  

\subsection{Shallow water}

We now move on to the one-dimensional shallow water equations, which consist of a conservation equation and momentum equation
\begin{align*}
\pd{h}{t} + \pd{hv}{x} &= 0\\
\pd{hv}{t} + \pd{\LRp{hv^2 + gh^2/2}}{x} + gh\pd{b}{x}&= 0.
\end{align*}
where $h$ is the water height, $v$ is the velocity, $g$ is the gravitational constant, and $b$ is bottom topography.  The entropy associated with this system is the total energy $e = k + p$, where $k, p$ are the kinetic and potential energies
\[
k = hv^2/2, \qquad p = gh^2/2 + ghb.
\]
The entropy variables can be derived from the entropy $e$
\[
q_1 = \pd{e}{h} = g(h+b)-v^2/2, \qquad q_2 = \pd{e}{hv} = v.
\]
Multiplying the continuity equation by $q_1$ and the momentum equation by $q_2$ recovers a statement of conservation of energy
\[
\pd{e}{t} + \pd{}{x}\LRp{\frac{hv^3}{2} + g(hv)(h+b)} = 0.  
\]
Like the Burgers' equation, the shallow water equations are stable under an appropriate split formulation\cite{gassner2016well, chen2017entropy}.  However, unlike Burgers' equation, the shallow water equations are not entropy stable with respect to the square entropy.  Additionally, a key difference between the shallow water equations and Burgers' equation is that, when discretizing with respect to the conservative variables $h, hv$, the entropy variables $v_1,v_2$ are not contained within the space $V_h$.  

%This conservation of energy does not hold at the discrete level for two reasons.  The first reason is because, under inexact quadrature, calculus identities such as the chain or product rule do not hold, and naive spatial formulations can not be manipulated into a form which is amenable to a proof of discrete entropy stability.  This issue can be remedied by the use of a split formulation (as shown in Gassner et al.) which is provably entropy stable under inexact quadrature.  
%
%The second more subtle reason why entropy conservation does not hold at the discrete level is because the entropy variables are not contained within the test space.  Moreover, the spatial part of the proposed discretization requires one discrete definition of the entropy variables for entropy stability, while the temporal part of the discretization requires another discrete definition of the entropy variables.  We address this by adopting a rescaling in the mass matrix on the left hand side involving both definitions of the discrete entropy variables.  Under an appropriate choice of basis and quadrature, the proposed approach recovers existing DG-SEM discretizations. 

\subsection{Compressible Euler equations}

We now test the proposed entropy conservative schemes on the one-dimensional compressible Euler equations, which correspond to the inviscid limit of the compressible Navier-Stokes equations
\begin{align*}
\pd{\rho}{t} + \pd{\LRp{\rho u}}{x} &= 0,\\
\pd{\rho u}{t} + \pd{\LRp{\rho u^2 + p }}{x} &= 0,\\
\pd{E}{t} + \pd{\LRp{u(E+p)}}{x} &= 0.
\end{align*}
We assume an ideal gas, such that the pressure satisfies the constitutive relation $p = (\gamma-1)\LRp{E - \frac{1}{2}\rho u^2}$, where $\gamma = 1.4$ is the ratio of specific heat for a diatomic gas.    

The choice of convex entropy for the Euler equations is non-unique \cite{harten1983symmetric}.  However, a unique entropy can be chosen by restricting to choices of entropy variables which symmetrize the viscous heat conduction term in the compressible Navier-Stokes equations \cite{hughes1986new}.  The entropy is given by
\[
U(\bm{u}) = -\frac{\rho s}{\gamma-1},
\]
where $s = \log\LRp{\frac{p}{\rho^\gamma}}$ is the physical specific entropy.  The entropy variables under this choice of entropy are then
\begin{align*}
q_1 = \frac{\gamma-s}{\gamma-1} - \frac{\rho u^2}{2p}, \qquad q_2 = \frac{\rho u}{p}, \qquad q_3 = -\frac{\rho}{p}.
\end{align*}
The reverse mapping is given by 
\[
d
\]

\note{Add Chandreshekar's fluxes \cite{chandrashekar2013kinetic}.}

Unlike the shallow water equations, these entropy conservative fluxes do not correspond to stable split formulations of the Euler equations.  Thus, the compressible Euler equations serve as a test of the flux differencing formulation, as entropy stability cannot be achieved through skew-symmetry.  

\section{Conclusions}

\section{Acknowledgments}

The author thanks Andrew Winters for several informative discussions.  Jesse Chan is supported by NSF awards DMS-1719818 and DMS-1712639.  

\bibliographystyle{unsrt}
\bibliography{dg}


\end{document}



\pdfoutput=1

\documentclass{article}
\usepackage{fullpage}
\usepackage{amsmath,amssymb,amsfonts}
\usepackage{bm}
%\usepackage{bbm}
%\usepackage{tikz}
%\usepackage[normalem]{ulem}
%\usepackage{hhline}
%\usepackage{graphicx}
%\usepackage{subfig}
\usepackage{color}
%\usepackage{doi}

%\renewcommand{\topfraction}{0.85}
%\renewcommand{\textfraction}{0.1}
%\renewcommand{\floatpagefraction}{0.75}
%
%
%\newcommand{\bbm}[1]{\mathbbm{#1}}
%\newcommand{\bs}[1]{\boldsymbol{#1}}
%\newcommand{\equaldef}{\stackrel{\mathrm{def}}{=}}
%
%
%\newcommand{\mb}[1]{\mathbf{#1}}
%\newcommand{\mbb}[1]{\mathbb{#1}}
%\newcommand{\mc}[1]{\mathcal{#1}}
%
%\renewcommand{\hat}{\widehat}
%\renewcommand{\tilde}{\widetilde}
\newcommand{\td}[2]{\frac{{\rm d}#1}{{\rm d}{\rm #2}}}
\newcommand{\pd}[2]{\frac{\partial#1}{\partial#2}}
%\newcommand{\pdn}[3]{\frac{\partial^{#3}#1}{\partial#2^{#3}}}
%\newcommand{\snor}[1]{\left| #1 \right|}
%\newcommand{\nor}[1]{\left\| #1 \right\|}
\newcommand{\LRp}[1]{\left( #1 \right)}
\newcommand{\LRs}[1]{\left[ #1 \right]}
%\newcommand{\LRa}[1]{\left\langle #1 \right\rangle}
%\newcommand{\LRb}[1]{\left| #1 \right|}
%\newcommand{\LRc}[1]{\left\{ #1 \right\}}
%\newcommand{\LRceil}[1]{\left\lceil #1 \right\rceil}
%\newcommand{\LRl}[1]{\left. \LRp{#1} \right|}
%\newcommand{\jump}[1] {\ensuremath{\llbracket#1\rrbracket}}
%\newcommand{\avg}[1] {\ensuremath{\LRc{\!\{#1\}\!}}}
%\newcommand{\Grad} {\ensuremath{\nabla}}
%\renewcommand{\d}{\partial}
\newcommand{\diag}[1]{{\rm diag}\LRp{#1}}

\newcommand{\note}[1]{{\color{blue}{#1}}}
\newcommand{\bnote}[1]{{\color{blue}{#1}}}
\newcommand{\rnote}[1]{{\color{red}{#1}}}


\newcommand{\LK}{L^2\LRp{D^k}}
\newcommand{\LdK}{L^2\LRp{\partial D^k}}
\newcommand{\Dhat}{\widehat{D}}
\newcommand{\Lhat}{L^2\LRp{\Dhat}}


\newcommand*\diff[1]{\mathop{}\!{\mathrm{d}#1}} % d in integrand

\begin{document}

%\maketitle

\note{We thank both Reviewers 1 and 2 for their feedback.  We describe steps taken to address reviewer comments and suggestions, which are described in the following response.  Revisions in the manuscript are also colored for ease of identification.  We hope these revisions improve the readability of this paper and its suitability for the audience of SISC.}

\section{Reviewer 1}

\begin{itemize}
\item The paper is well-written for the most part (authors should adjust a few grammatical typos throughout their paper), and is interesting. 
\item It would be good if the authors would add a few more numerical tests with model problems in domains with complex geometry where curved meshes are actually needed (at this point, authors considered either rectangular or parallelepiped domain).
\end{itemize}

\section{Reviewer 2}

\begin{itemize}
\item Page 7, eqn (3.8): why this equation holds? Can you add some comment here?

\note{Certainly.  We've added a continuous description of the equation, and have interpreted it as a quadrature approximation of a specific variational problem.}

\item Page 10, Theorem 3.4: In the proof, this leads to a extra coefficient 1/2 in front of the second term of the equation below (3.14).
\item In Theorem 3.4, it presents an semi-discrete entropy conservation for a two-element mesh. How to extend this to general case with multiple elements?
\item In the statement, boundary condition is not considered. How to utilize this theorem to solve the initial-boundary value problem numerically?
\item A pseudo code showing how to solve 1D conservation law would be very helpful
\item Page 19, Section 5.3: this problem is solved on the triangular mesh. How to generalized the proposed scheme from quadrilateral and hexahedral meshes to triangular mesh?
\item Page 4, line 152: Does the matrix size of $V_f$ equal to $(N+1) \times 2$ or $2 \times (N+1)$?
\item Page 14. line 495: $\bm{u} = (\bm{u}_1, . . . , \bm{u}_d)$. Is $\bm{u}_i$ a vector or a scalar?  
\item Page 15, line 518-522: $u$ and $v$ here represent the velocity in $x$ and $y$-direction, respectively. These conflict with the notations in previous sections, where, $u$ is the conservative variable and $v$ is the entropy variable. Definition of $\{\{\cdot\}\}$ should be added here, too.
\end{itemize}

\end{document}

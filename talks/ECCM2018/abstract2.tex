
\documentclass[a4paper]{article} %
\usepackage{graphicx,amssymb} %

\textwidth=15cm \hoffset=-1.2cm %
\textheight=25cm \voffset=-2cm %

\pagestyle{empty} %

\date{} %

\def\keywords#1{\begin{center}{\bf Keywords}\\{#1}\end{center}} %



% Please, do not change any of the above lines




\begin{document}

% Type down your paper title
\title{Efficient explicit solvers for multipatch discontinuous Galerkin isogeometric analysis}

% Authors
\author{Jesse Chan, \\ %
       Rice University (USA) \\ \\ % Affiliation 1
       John A. Evans, \\ %
       UC Boulder (USA) \\ \\ % Affiliation 1
       \tt{jesse.chan@rice.edu} % Only one corresponding e-mail
       }%


\maketitle

\thispagestyle{empty}

% The abstract

\begin{abstract}
In this talk, we present explicit solvers for time-dependent hyperbolic problems on curved geometries using isogeometric analysis and spline-based finite element methods.  Explicit finite element solvers require mass matrix inversions at each time-step.  For affine geometries in two and three dimensions, isogeometric mass matrices possess a Kronecker product structure, which can be exploited to reduce the application of the inverse mass matrix to the application of one-dimensional inverse mass matrices.  However, this Kronecker structure is lost for curved domains due to the introduction of a spatially varying weight into the isogeometric mass matrix.  The inversion of a large dense mass matrix can be avoided by approximating the inverse mass matrix using techniques such as mass lumping; however, it can be difficult to maintain both energy stability and high order accuracy under such approaches.

We present a simple ``weight-adjusted'' approximation for the inverse of isogeometric mass matrices on a single curved patch.  This approximation can be applied in a Kronecker-like manner using one-dimensional spline matrices, while maintaining provable energy stability and high order accuracy.  We extend this approach to multiple geometric patches using a multi-patch discontinuous Galerkin isogeometric analysis (DG-IGA) approach.  We also describe advantages offered by isogeometric methods over $C^0$ and discontinuous high order finite element discretizations when paired with explicit time integration on curvilinear geometries.  Numerical results in one, two, and three dimensions confirm theoretical results and illustrate the advantages of the proposed methods.  

%We present a class of spline finite element methods for time-domain wave propagation which are particularly amenable to explicit time-stepping. The proposed methods utilize a discontinuous Galerkin discretization to enforce continuity of the solution field across geometric patches in a multi-patch setting, which yields a mass matrix with convenient block diagonal structure. Over each patch, we show how to accurately and efficiently invert mass matrices in the presence of curved geometries by using a weight-adjusted approximation of the mass matrix inverse. This approximation restores a tensor product structure while retaining provable high order accuracy and semi-discrete energy stability. We also estimate the maximum stable timestep for spline-based finite elements and show that the use of spline spaces result in less stringent CFL restrictions than equivalent piecewise continuous or discontinuous finite element spaces. Finally, we explore the use of optimal knot vectors based on $L^2$ $n$-widths. We show how the use of optimal knot vectors can improve both approximation properties and the maximum stable timestep, and present a simple heuristic method for approximating optimal knot positions. Numerical experiments confirm the accuracy and stability of the proposed methods.

\end{abstract}

\keywords{Isogeometric analysis, high order, mass matrix, explicit time-stepping} % Write down at least 3 Keywords

\begin{thebibliography}{1}
\bibitem{key} Chan, Jesse, and John A. Evans. \emph{Multi-patch discontinuous Galerkin spline finite element methods for time-domain wave propagation.} arXiv preprint arXiv:1708.02972 (2017).  Submitted to CMAME, in revision.
\bibitem{mpiga} Langer, Ulrich, Angelos Mantzaflaris, Stephen E. Moore, and Ioannis Toulopoulos. \emph{Multipatch discontinuous Galerkin isogeometric analysis.} In Isogeometric Analysis and Applications 2014, pp. 1-32. Springer, Cham, 2015.  	
\end{thebibliography}



% \section{Introduction}






\end{document}
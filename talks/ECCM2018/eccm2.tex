\documentclass[12pt]{eccm-ecfd_abstract}

\usepackage{graphicx}
\usepackage{amsmath}
\usepackage{amsfonts}
\usepackage{amssymb}
\usepackage{verbatim}

\title{Efficient explicit solvers for multipatch discontinuous Galerkin isogeometric analysis}

\author{Jesse Chan$^{1}$ and John A. Evans$^{2}$}

\address{$^{1}$ Rice University, 6100 Main St MS-134, Houston, TX 77005,\\{jesse.chan@rice.edu}, http://www.caam.rice.edu/\textasciitilde jchan985
\and $^{2}$ University of Colorado Boulder, 1111 Engineering Drive, Boulder, CO 80309,\\john.a.evans@colorado.edu, {spot.colorado.edu/\textasciitilde joev9679}}
%\and
%$^{3}$ Affiliation, Postal Address, E-mail address and URL}

\begin{document}

\noindent {\bf Keywords}: {\it Isogeometric analysis, high order, mass matrix, explicit time-stepping}
\vskip0.5cm

In this talk, we present explicit solvers for time-dependent hyperbolic problems on curved geometries using isogeometric analysis and spline-based finite element methods.  Explicit finite element solvers require mass matrix inversions at each time-step.  For affine geometries in two and three dimensions, isogeometric mass matrices possess a Kronecker product structure, which can be exploited to reduce the application of the inverse mass matrix to the application of one-dimensional inverse mass matrices.  However, this Kronecker structure is lost for curved domains due to the introduction of a spatially varying weight into the isogeometric mass matrix.  The inversion of a large dense mass matrix can be avoided by approximating the inverse mass matrix using techniques such as mass lumping; however, it can be difficult to maintain both energy stability and high order accuracy under such approaches.

We present a simple ``weight-adjusted'' approximation for the inverse of isogeometric mass matrices on a single curved patch.  This approximation can be applied in a Kronecker-like manner using one-dimensional spline matrices, while maintaining provable energy stability and high order accuracy.  We extend this approach to multiple geometric patches using a multi-patch discontinuous Galerkin isogeometric analysis (DG-IGA) approach.  We also describe advantages offered by isogeometric methods over $C^0$ and discontinuous high order finite element discretizations when paired with explicit time integration on curvilinear geometries.  Numerical results in one, two, and three dimensions confirm theoretical results and illustrate the advantages of the proposed methods.  

\begin{thebibliography}{99}
\bibitem{key} Chan, Jesse, and John A. Evans. "Multi-patch discontinuous Galerkin isogeometric analysis for wave propagation: Explicit time-stepping and efficient mass matrix inversion." Computer Methods in Applied Mechanics and Engineering 333 (2018): 22-54.
%\bibitem{mpiga} Langer, Ulrich, Angelos Mantzaflaris, Stephen E. Moore, and Ioannis Toulopoulos. \emph{Multipatch discontinuous Galerkin isogeometric analysis.} In Isogeometric Analysis and Applications 2014, pp. 1-32. Springer, Cham, 2015.  	
\end{thebibliography}


%\begin{thebibliography}{99}
%\bibitem{Chan}  
%\end{thebibliography}

\end{document}



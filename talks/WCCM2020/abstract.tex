
\documentclass[a4paper]{article} %
\usepackage{graphicx,amssymb} %

\textwidth=15cm \hoffset=-1.2cm %
\textheight=25cm \voffset=-2cm %

\pagestyle{empty} %

\date{} %

\def\keywords#1{\begin{center}{\bf Keywords}\\{#1}\end{center}} %



% Please, do not change any of the above lines




\begin{document}

% Type down your paper title
\title{Extensions of entropy stable nodal discontinuous Galerkin schemes}

% Authors
\author{Jesse Chan, \\ %
       Rice University (USA) \\ \\ % Affiliation 1
       \tt{jesse.chan@rice.edu} % Only one corresponding e-mail
       }%


\maketitle

\thispagestyle{empty}

% The abstract

\begin{abstract}
High order discontinuous Galerkin (DG) methods provide improved accuracy and low numerical dispersion/dissipation for simulations of nonlinear conservation laws.  However, these methods also tend to suffer from instability in practice, requiring filtering, limiting, or artificial dissipation to prevent solution blow up.  \textit{Entropy stable} nodal DG methods based on summation-by-parts (SBP) operators and flux differencing address this instability by ensuring satisfaction of a semi-discrete entropy inequality.  In this talk, we will present extensions of entropy stable DG methods to more general nodal points, which provide improved accuracy on warped and non-conforming meshes.  
\end{abstract}

\keywords{High order, discontinuous Galerkin, entropy stability} % Write down at least 3 Keywords



% \section{Introduction}




\end{document}